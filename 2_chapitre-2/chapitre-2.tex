\documentclass[11pt]{article}

% Informations

% Modules 
\usepackage[greek,french,english]{babel}
\usepackage{graphicx}                       % Gestion des images
\usepackage{caption}                        % Gestion des titres
\usepackage{appendix}                       % Gestion de l'annexe
\usepackage[utf8]{inputenc}                 % Encodage du texte
\usepackage{multicol}                       % Gestion des multi-colonnes des tableaux
\usepackage{booktabs}                       % Importation des traits horizontaux des tableaux
\usepackage{siunitx}                        % Gestion des unités
\usepackage{mwe,lipsum}                     % Modules de Minimal Working Examples
\usepackage{amsmath,amssymb,amsbsy}         % Ajout d'options dans le mode 'math'
\usepackage{todonotes}                      % Ajout des notes en marge
\usepackage{mathtools}                      % Autres outils pour le mode 'math'
\usepackage{hyperref}            % Gestion des hyper-liens (internes et url)
\usepackage[a4paper]{geometry}                       % Options de mise en page
\usepackage{listings}
\usepackage{color,xcolor}
\usepackage{csquotes}
\usepackage{textcomp}
\usepackage{fancyhdr}
\usepackage[T1]{fontenc}
\usepackage{lato}
\usepackage{titling}
\usepackage{datetime}
\usepackage[version=4]{mhchem}
\usepackage{authblk}
\usepackage{enumitem}
\usepackage{tablefootnote}
\usepackage{cases}
\usepackage{bbm}
\usepackage{lmodern}
\usepackage{empheq}
\usepackage{stmaryrd}


\setitemize{itemsep=0pt}
\setcounter{MaxMatrixCols}{20}

% Police d'écriture
\renewcommand\familydefault{\sfdefault}

% Définition de couleurs
% \definecolor{Bleu_ENSPS}{RGB}{0,119,139}
\definecolor{CIRED_blue}{RGB}{6,100,110}
\hypersetup{
    hidelinks,
    }

% Options de biliographie 
\usepackage[style=authoryear,giveninits=true,sorting=nty,maxcitenames=1]{biblatex}
% \usepackage[backend=biber, bibstyle=ieee, citestyle=numeric-comp,
%   sorting=none, labeldateparts,
%   maxbibnames=99, maxcitenames=2, mincitenames=1]{biblatex} 
\DefineBibliographyExtras{french}{\restorecommand\mkbibnamefamily}

\AtEveryBibitem{%
  \clearlist{language}%
  \clearlist{urldate}
  \clearlist{url}
  \clearfield{month}
  \clearfield{day}
  \clearfield{note}
}
\DeclareFieldFormat{url}{}
\DeclareFieldFormat{urldate}{}

\DeclareFieldFormat{journaltitle}{\textit{#1}}
\DeclareFieldFormat{title}{#1}

\setlength\bibitemsep{\itemsep}
\AtEveryCite{\color{CIRED_blue}}

\addbibresource{/home/amounier/Documents/Bibliographie/bibliographie_bib.bib}
% \bibliography{biblatex-examples.bib}

% All name in hyperlink (cite biblatex)
\makeatletter
\let\abx@macro@citeOrig\abx@macro@cite
\renewbibmacro{cite}{%
   \bibhyperref{%
   \let\bibhyperref\relax\relax%
   \abx@macro@citeOrig%
   }%
}
\let\abx@macro@textciteOrig\abx@macro@textcite
\renewbibmacro{textcite}{%
   \bibhyperref{%
   \let\bibhyperref\relax\relax%
   \abx@macro@textciteOrig%
   }%
}%
\makeatother

% Options de format pour les unités (SIUnitX package)
\sisetup{
    detect-all,
    locale                  = UK,
    sticky-per,
    inter-unit-product      = {.},
    per-mode                = reciprocal-positive-first,
}

\DeclareSIUnit\octet{o}
\DeclareSIUnit\watthour{Wh}
\DeclareSIUnit\year{yr}
\DeclareSIUnit\hab{inhab}
\DeclareSIUnit\volume{V}

%Options de largeur de marges, verticales et horizontales
\geometry{hmargin=3cm,vmargin=2cm}
\setlength{\parindent}{7mm}

% Mise en page
% \pagestyle{fancy}
% \setlength{\headheight}{14pt}
% \renewcommand\headrulewidth{0.5pt}
% \renewcommand\footrulewidth{0.5pt}
% \fancyhead[C]{\thedate}
% \fancyhead[R]{}
% \fancyhead[L]{}
% \fancyhead[R]{\rightmark}

% Styles équations
%\numberwithin{equation}{section}

\makeatletter
\renewcommand\p@figure{\small{Figure~\@ }}
\makeatother

\makeatletter
\renewcommand\p@table{\small{Table~\@ }}
\makeatother

%Autres commandes
\addto\captionsfrench{
  \renewcommand{\contentsname}%
    {Sommaire}%
}

% Keywords command
\providecommand{\keywords}[1]
{
  \small    
  \textbf{\textit{Keywords~--}} #1
}

% \renewcommand{\paragraph}[1]{\paragraph{#1}\mbox{}\\}

% Titre ---------------------------------------
\date{\today}
\title{Interactions between summer and winter thermal comfort, effects of climate change on optimal renovation actions}

\author[1,3,4]{André Mounier}
\author[2,3]{Louis-Gaëtan Giraudet}
\author[4]{Philippe Drobinski}
\affil[1]{\small{Agence de l'environnement et de la maîtrise de l'énergie (ADEME), Angers, France}}
\affil[2]{\small{ENPC - Institut Polytechnique de Paris, Champs-sur-Marne, France}}
\affil[3]{\small{CIRED -- ENPC, AgroParisTech, EHESS, Cirad, CNRS, Nogent-sur-Marne, France}}
\affil[4]{\small{LMD -- IPSL, École Polytechnique - IPP, ENS - PSL , Sorbonne Université, CNRS, Palaiseau, France}}

\begin{document}


\maketitle

% Contenu -------------------------------------

% Méthode :
% https://www.nature.com/documents/nature-summary-paragraph.pdf
\begin{abstract}
    Energy efficiency building renovation, particularly through thermal insulation, is a key factor in the transformation of the building stock to reduce greenhouse gas emissions and adapt dwellings to future climates. Thermal renovation of buildings is a particularly costly operation that rarely pays off for the person carrying out the work. In France, many renovation projects are financed in part by the state, and this funding involves targeting the most effective works. However, this efficiency is only measured in terms of heating needs, while the critical nature of the inadequacy of housing for future heat increases year on year.  Taking account of climate change, its future evolution, and the need to adapt homes will therefore influence the optimum renovations to target as a priority now. Here we show that the displacement of the optimum differs according to the type of building and the intensity of warming, and according to the type of renovation work. This study presents a first way of considering dynamic, energy and economic optimisations, with RC modelling of building typologies. The conclusions are different depending on the type of action carried out: for example, the insulation of opaque walls does have an antagonistic effect on heating and cooling needs (but this is particularly visible in the coldest meteorological years and without night-time natural over-ventilation), but total energy needs are always decreasing. Thus, inter-annual variations in typology and weather play a decisive role in defining the optimum. The results call into question the selection criteria for subsidised renovations, as well as the renovations carried out in practice in France. The question of the interaction between summer and winter comfort has been raised in official reports in France, because adaptation is underdeveloped, and the majority of measures are focused on winter thermal comfort. A deliberately provocative question could be: ‘Do we need to renovate buildings on a massive scale if the French climate warms up significantly between now and the end of the century? The answer is yes, but not for all buildings or all regions in the same way. 
\end{abstract}

\keywords{Thermal insulation, Heating and cooling needs, Adaptation, Mitigation}

\clearpage
\tableofcontents

\clearpage

\section{Introduction}
\label{sec:intro}

consommations des batiments dans le monde et en France (\cite{unep_2023_2024}, \cite{sdes_chiffres_2023}, \cite{sdes_consommation_2023})

changement climatique en France (physique et scenarios officiels) (\cite{ipcc_climate_2021}, \cite{ouzeau_heat_2016}, \cite{ministere_de_la_transition_ecologique_trajectoire_2023})

consommations futures des batiments (\cite{larsen_climate_2020},\cite{moreau_evaluation_2023},\cite{filahi_projections_2024}, \cite{tao_uncertainty_2024})

inadaptation du parc aux temperatures futures (\cite{cour_des_comptes_laction_2024}, \cite{i4ce_vagues_2024})

rentabilité et cout des renovations (\cite{ademe_renovation_2019}, \cite{i4ce_trajectoires_2023}, \cite{giraudet_analyse_2024})

critères de soutiens aux travaux d'isolation (\cite{france_strategie_dispositif_2024}, \cite{coulaud_maprimerenov_2024}, \cite{anah_aides_2024-1}, \cite{dagostino_impact_2024})


\clearpage
\section{RC analogy modelling}
\label{sec:rc}

    \subsection{RC analogy and computation} % (fold)
    \label{sub:rc_analogy_and_computation}

        \subsubsection{RC analogy} % (fold)
        \label{ssub:rc_analogy}

        In the 19\textsuperscript{th} century, the idea emerged that \enquote{electricity behaves like heat} (\cite{bolmont_evolution_2003}). The laws of heat were well known by then, thanks in particular to the work of Fourier (\cite{fourier_theorie_1822}). Ohm developed the analogy and formalised it to create his first laws on the behaviour of electricity (\cite{ohm_galvanische_1827}). Although the justification of the analogy is more delicate for the concepts of thermal capacity and electrical capacity, the equivalences of the physical quantities can be summarised in \ref{tab:analogyrc} (\cite{bolmont_evolution_2003}). Finally, it is possible to apply this analogy in the other way round and model a building as an electrical network in order to use the Kirchhoff's circuit laws and easily model variations in temperature and heat flow. \\

        \begin{table}[ht]
          \centering
          \small
          \caption{\label{tab:analogyrc} Table of thermal-electrical analogies.}
          \begin{tabular}{lcl}
            \toprule
            Heat & & Electricity\\
            \midrule
            Temperature difference & $\leftrightarrow$ & Voltage potential\\
            Heat flow & $\leftrightarrow$ & Electric current\\
            Thermal resistance & $\leftrightarrow$ & Electrical resistance\\
            Thermal capacity & $\leftrightarrow$ & Electrical capacity\\
            \bottomrule
          \end{tabular}
        \end{table}

        This electrical/thermal analogy has long been used for thermal modelling of buildings, particularly for simplified building models. The numerical resolution of RC models with few resistors and capacitors to model the energy consumption of buildings has existed for over 40 years (\cite{francis_methode_1982}, \cite{madsen_estimation_1995}, \cite{fraisse_development_2002}). An example of a small RC model is detailed in paragraph 2 to illustrate the calculation method used in this study. Simplified RC models are also used in the ISO-13790 standard (\cite{iso_iso_2006}), with five resistors and an equivalent capacitor, to which most researchers add a second capacitor to take account of the inertia of the internal thermal zone (\cite{parc_etude_2014}, \cite{marty-jourjon_identifiability_2022}). \\

        Using these simplified models, it is possible to increase the number of electronic components to take into account more and more phenomena and different inertias, such as the inertia of heating systems ( \cite{bacher_identifying_2011}) or air infiltration (\cite{reynders_quality_2014}). Other researchers are even combining several simplified models to produce multi-zone modelling, enabling several different temperatures to be modelled within the building (\cite{belazi_thermal_2022}, \cite{vallianos_automatic_2022}, \cite{balali_energy_2023}). Finally, by combining the two aspects, other researchers have sought to represent as accurately as possible each component of the dwelling and the associated heat flows in order to obtain a detailed RC model (\cite{wang_development_2019}, \cite{cui_model_2022}). The modelling used here is intermediate, with a detailed RC model with a single thermal zone in the building, including multi-family dwellings. \\

        This last point introduces the limits of this method. In general, RC models are used as grey-box parametric models whose R and C values are the result of calibration on real measured temperatures and consumptions. Thus, at the end of the calibration, the physical interpretation of the parameter values is not guaranteed, depending on the structure of the RC network. In our case, we are applying RC models to building typologies which, by definition, have no actual consumption. It is therefore not possible to calibrate the parameters, and we use the RC model as a conventional physical model. It is therefore certain that the results obtained cannot be used to forecast real energy consumption on a building scale. In this study, we use the modelling results in relative terms for conventional energy needs, to compare the effects of one or more renovation actions on changes in heating and cooling needs.  

        % subsubsection rc_analogy (end)

        \subsubsection{Model construction and resolution} % (fold)
        \label{ssub:model_construction}

        \paragraph{Simplified RC model for illustration}\mbox{}\\ % (fold)
        \label{par:simplified_rc_model}

        In order to illustrate and explain the computation and resolution methods used in this study, we detail the solving in the context of a simplified RC model, inspired by examples proposed in the literature (\cite{madsen_estimation_1995}, \cite{bacher_identifying_2011}, \cite{rouchier_solving_2018}). This model considers only two thermal inertias, that of the wall and that of the indoor air. The internal heat sources are internal heat gains (heating system) and internal solar radiation coming through the window. In this example, solar radiation reaching the external facade are not considered. Finally, we consider three thermal resistances: two for the wall (outside and inside) and one for the window. The modelled temperatures are the external temperature, the wall internal temperature and the internal temperature. \ref{fig:RClight} illustrates this example and shows the equivalent RC diagram. 

        \begin{figure}[ht]
            \centering
            \includegraphics[width=0.49\columnwidth]{figures/R3C2_diagram.drawio.png}
            \includegraphics[width=0.49\columnwidth]{figures/R3C2.pdf}
            \caption{\label{fig:RClight} Diagram of a building and the associated simplified RC model}
            % \begin{quote}
            %     \vspace{-2mm}
            %     \small\noindent
            %     The diagram and the nspired from \textcite{madsen_estimation_1995}.
            % \end{quote}
         \end{figure}

        By applying Kirchhoff's circuit laws to our simplified model, the following two-equation system \eqref{eq:eq1rclight} is obtained: 
        % The RC model shown in the \ref{fig:RClight} provides a simplified representation of the general model described below (\ref{par:general_rc_model}). The various thermal and energy variables represented are as follows:

        \begin{subequations}\label{eq:eq1rclight}
            \begin{empheq}[left=\empheqlbrace]{align}
            C_1\frac{\mathrm{d}T_w}{\mathrm{d}t} &= \frac{1}{R_1}(T_e-T_w) - \frac{1}{R_2}(T_w-T_i)\\
            C_2\frac{\mathrm{d}T_i}{\mathrm{d}t} &= \Phi_{s,\mathrm{int}} + q + \frac{1}{R_2}(T_w-T_i) + \frac{1}{R_3}(T_e-T_i)
            \end{empheq}            
        \end{subequations}

        \noindent
        With,
        $$
        \begin{dcases}
          T_e&:\text{ External temprature (\SI{}{\celsius})} \\
          T_w&:\text{ Wall internal temperature (\SI{}{\celsius})} \\
          T_i&:\text{ Internal temperature (\SI{}{\celsius})} \\
        \end{dcases}\quad\text{and}\quad
        \begin{dcases}
          R&:\text{ Thermal resistance (\SI{}{\kelvin\per\watt})} \\
          C&:\text{ Thermal capacity (\SI{}{\joule\per\kelvin})} \\
          \Phi_{s,\mathrm{int}}&:\text{ Internal solar flux (\SI{}{\watt})} \\
          q&:\text{ Internal energy gains (\SI{}{\watt})} \\
        \end{dcases}
        $$

        The system of equations can then be rewritten in matrix form \eqref{eq:matrix}, which is referred to as the state equation. Where $\mathbf{x}$ is the vector of inertial temperatures and $\mathbf{u}$ is the vector of heat source variables. 
        \begin{equation}\label{eq:matrix}
          \frac{\mathrm{d}}{\mathrm{d}t}\left(\begin{bmatrix}
            T_i\\
            T_w
          \end{bmatrix}\right) = \mathbb{A} \cdot \underbrace{\vphantom{\begin{bmatrix}
            T_e\\
            \Phi_{s,\mathrm{int}}\\
            q
          \end{bmatrix}}\begin{bmatrix}
            T_i\\
            T_w
          \end{bmatrix}}_{\mathbf{x}} + \mathbb{B}\cdot\underbrace{\begin{bmatrix}
            T_e\\
            \Phi_{s,\mathrm{int}}\\
            q
          \end{bmatrix}}_{\mathbf{u}}
        \end{equation}
        \noindent
        Where,
        $$
        \mathbb{A}  = \begin{bmatrix}
            -\frac{1}{R_2 C_2} - \frac{1}{R_3 C_2} & \frac{1}{R_2 C_2}\\
            \frac{1}{R_2 C_1} & -\frac{1}{R_1 C_1} - \frac{1}{R_2 C_1}\\
          \end{bmatrix}
        \quad\text{and}\quad
        \mathbb{B}  = \begin{bmatrix}
            \frac{1}{R_3 C_2} & \frac{1}{C_2} & \frac{1}{C_2}\\
            \frac{1}{R_1 C_1}  & 0 & 0\\
          \end{bmatrix}
        $$

        The time domain is then discretised into a succession of moments separated by a time step $\delta$. If and only if the matrices $\mathbb{A}$ and $\mathbb{B}$ are constant over time, the vector of inertial temperatures $\mathbf{x}$ at time $t+\delta$ can be written as (\cite{brogan_modern_1991}):
        \begin{equation}\label{eq:xtdelta}
            \mathbf{x}_{t+\delta} = \exp\left(\mathbb{A}\delta\right)\mathbf{x}_{t} + \int_t^{t+\delta} \exp\left(\mathbb{A}(t+\delta-\tau)\right) \mathbb{B}\mathbf{u}(\tau)\mathrm{d}\tau
        \end{equation}
        
        \noindent
        With,
        $$
        \begin{dcases}
          \exp \mathbb{A}= \sum_{k=0}^\infty \frac{1}{k!}\mathbb{A}^k\\
          \mathbf{u}(\tau)= \mathbf{u}_t + \frac{\tau-t}{\delta}\left(\mathbf{u}_{t+\delta}-\mathbf{u}_t\right)\\
        \end{dcases}
        $$

        With a sufficiently small time step $\delta$ and a sufficiently gently varying $\mathbf{u}$ vector, it is possible to neglect the second term in the expression of $\mathbf{u}(\tau)$. In this case, equation \eqref{eq:xtdelta} can be rewritten as (\cite{seem_transfer_1989}, \cite{madsen_estimation_1995}): 
        \begin{equation}\label{eq:xtfg}
            \mathbf{x}_{t+\delta} = \mathbb{F}\cdot\mathbf{x}_{t} + \mathbb{G}\cdot\mathbf{u}_{t}
        \end{equation}
        
        \noindent
        With,
        $$
        \begin{dcases}
          \mathbb{F} = \exp\left(\mathbb{A}\delta\right)\\
          \mathbb{G}= \mathbb{A}^{-1}\left(\mathbb{F}-\mathbb{I}\right)\mathbb{B}&\text{, with $\mathbb{I}$ the identity matrix}\\
        \end{dcases}
        $$

        This integration method makes the temperature calculation at each time step very inexpensive in terms of computation time. The most costly step is the computation of the $\mathbb{F}$ and $\mathbb{G}$ matrices, which is performed only once at the beginning of the resolution because the $\mathbb{A}$ and $\mathbb{B}$ matrices are constant over time. Therefore, all time-dependent variables (such as solar radiation and heat flow due to ventilation) must be included in the $\mathbf{u}$ vector. 
        
        
        % paragraph simplified_rc_model (end)

        \paragraph{General RC model}\mbox{}\\ % (fold)
        \label{par:general_rc_model}
        
        In the complete model, the thermal envelope of residential buildings is not modelled using a single wall. The building types used in this study (\ref{sub:tabula_typologies}) are represented by six different surfaces: the 4 walls (each with a different orientation), the roof and the floor. Only one thermal zone is considered per building, even for multi-family dwellings. Gains from solar radiation are taken into account on external walls and inside the building. All the resistances, inertias and sources of heat flow are shown in \ref{fig:rc_mod}. Depending on the type of building, certain parts of the RC network may be short-circuited. This is the case when buildings do not have a basement or when attics are converted. \\

        \begin{figure}[ht]
            \centering
            \includegraphics[width=0.99\columnwidth]{figures/genmod.pdf}
            \caption{\label{fig:rc_mod} Diagram of the general RC model.}
            \begin{quote}
                \vspace{-2mm}
                \small\noindent
                In a similar way to the simplified diagram shown above (\ref{fig:RClight}), each branch of the equivalent electrical diagram can be interpreted. Firstly, the voltage generator on the left sets the outside temperature. The part on the right indicates the internal flows and inertia. From top to bottom, the branches linking the outside to the inside represent: air infiltration, exchanges through the upper horizontal surface (roof and ceiling), exchanges through the four walls, and exchanges through the floor directly above the ground, the temperature of which ($T_g$) is directly integrated into the model (\ref{par:ground_temperature}). 
              \end{quote}
        \end{figure} 

        At each time step, we explicitly calculate $\mathbf{x}_{t+\delta}$ of equation \eqref{eq:xtfg}, by updating the vector $\mathbf{u}_{t}$. In particular, the power of the internal emitters $\Phi_{hc}$, the value of which is a function of the indoor temperature ($T_i$) and the set-point temperature for heating ($T_{sh}$) and cooling ($T_{sc}$) \eqref{eq:power_hc}, with $\pm$\SI{1}{\celsius} tolerance for smoothing response. 

        \begin{subequations}\label{eq:power_hc}
            \begin{empheq}[left=\empheqlbrace]{align}
            \Phi_h &= 
            \begin{cases}
                P_{h,\mathrm{max}} & \text{ if } T_i < T_{sh} - 1\\
                P_{h,\mathrm{max}}~\left(1+\left((T_{sh} - 1) - T_i\right)/2\right) & \text{ if } T_i \in \left[T_{sh} - 1, T_{sh} + 1\right[\\
                0 & \text{ else.}
            \end{cases}\\
            \Phi_c &=
            \begin{cases}
                0 & \text{ if } T_i \leq T_{sc} - 1\\
                P_{c,\mathrm{max}}~\left(T_i-(T_{sc} - 1)\right)/2 & \text{ if } T_i \in \left]T_{sc} - 1, T_{sc} + 1\right]\\
                P_{c,\mathrm{max}} & \text{ else.}
            \end{cases}\\
            \Phi_{hc} &= \Phi_h - \Phi_c
            \end{empheq}            
        \end{subequations}

        % paragraph general_rc_model (end)
        % subsubsection model_construction (end)

        \subsubsection{Modelling of physics phenomena} % (fold)
        \label{ssub:model_computation}
        
            \paragraph{Materials}\mbox{}\\ % (fold)
            \label{par:materials}
            
            To simplify the analysis, we consider only a small number of materials and use only three physical quantities: density, thermal conductivity and specific heat capacity. All the materials used are described in \ref{tab:materials} (\cite{ministere_de_la_transition_ecologique_regles_2017}, \cite{castagnede_tables_2020}, \cite{sassine_investigation_2022}).


            \begin{table}[ht]
            \centering
            \caption{\label{tab:materials} Table of material characteristics.}
            \small
            % \resizebox{\columnwidth}{!}{
            \begin{tabular}{llccc}
                \toprule
                & & Density & Thermal conductivity & Thermal capacity \\
                & & (\SI{}{\kilo\gram\per\cubic\meter}) & (\SI{}{\watt\per\meter\kelvin}) & (\SI{}{\joule\per\kilo\gram\kelvin}) \\
                \midrule
                Structure & Wood & 600 & 0.150 & 1880 \\
                 & Stone & 2300 & 1.350 & 1000 \\
                 & Brick & 1650 & 0.640 & 1000 \\
                 & Concrete & 2100 & 1.650 & 1000 \\
                 & Breeze block & 1185 & 0.900 & 1080 \\
                 & Cellular concrete & 550 & 0.200 & 1000 \\
                 & Insulating brick & 700 & 0.250 & 900 \\
                \midrule
                Insulation & Polystyrene & 16 & 0.044 & 1450 \\
                 & Mineral wool & 25 & 0.041 & 1030 \\
                 & Vegetal wool & 90 & 0.078 & 1000 \\
                \bottomrule
                \end{tabular}
            % }
            % \begin{quote}
            %     \vspace{-2mm}
            %     \small\noindent
            %     The time periods under the climate zone names correspond to the time range available for the weather station closest to the location defined for the climate region (i.e. the central prefecture (\ref{fig:zcl})). 
            % \end{quote}
        \end{table}
            % paragraph materials (end)

            \paragraph{Thermal conduction}\mbox{}\\ % (fold)
            \label{par:thermal_conduction}
            
            The first physical phenomenon to be represented is heat conduction through a homogeneous element of thickness $e$ and thermal conductivity $\lambda$, assuming one-dimensional transfer. The heat transfer area on either side of this surface is referred to as $S_d$ \eqref{eq:R_cond}.

            \begin{equation}\label{eq:R_cond}
                R = \frac{e}{\lambda S_d} \quad \text{in \SI{}{\kelvin\per\watt}}
            \end{equation}

            Other assumptions concerning thermal conduction are applied. Firstly, the existence of thermal bridges is only considered in the case of internal insulation. To avoid the complexity of calculating thermal bridges (\cite{mao_dynamic_1997}, \cite{martin_problems_2011}, \cite{ministere_de_la_transition_ecologique_regles_2017-1}), we use the same technique as in the TABULA project methodology: thermal bridges are simply included as a \SI{27}{\percent} reduction in the efficiency of internal insulation (\cite{pouget_consultants_batiments_2015}). Secondly, some walls are considered adiabatic in our model. This is the case for shared walls in the case of terraced houses and collective dwellings, where applicable. In this configuration, the solution implemented to maintain the same RC model and the same resolution is to replace the outdoor temperature with the indoor temperature at each resolution time step. 

            % paragraph thermal_conduction (end)

            \paragraph{Convective and radiative heat transfer}\mbox{}\\ % (fold)
            \label{par:convective_and_radiative_heat_transfer}

            When a wall is in contact with the air, there is heat exchange in the form of convection and radiation (in the infra-red wavelengths). The radiative and convective exchange coefficients can then be written in the form of thermal resistances, which differ depending on whether the contact is outside or inside the envelope. In the interior case, the convective-radiative thermal resistance is described in equation 5, depends on the exchange surface $S_d$ and the direction of heat exchange (upward $\uparrow$, horizontal $\leftrightarrow$, downward $\downarrow$) (\cite{ministere_de_la_transition_ecologique_regles_2017-2}). 

            \begin{equation}\label{eq:Rih_horiz}
                R_{ih} = \frac{R_{si}}{S_d} \quad \text{in \SI{}{\kelvin\per\watt}, where}\quad
                \begin{dcases}
                    R_{si,\uparrow} = \SI{0.15}{\square\meter\kelvin\per\watt}\\
                    R_{si,\leftrightarrow} = \SI{0.13}{\square\meter\kelvin\per\watt}\\
                    R_{si,\downarrow} = \SI{0.17}{\square\meter\kelvin\per\watt}
                \end{dcases}
            \end{equation}

            In the case of convective-radiative exchange towards the outside, the two phenomena are more clearly distinguished \eqref{eq:Reh}. The convective exchange coefficient ($h_c$) is a function of wind speed, which is assumed to be constant at \SI{5}{\meter\per\second}. And the radiative exchange coefficient ($h_r$) is a function of the average temperature of the environment, considered constant at \SI{10}{\celsius} or \SI{283.15}{\kelvin} (\cite{ministere_de_la_transition_ecologique_regles_2017-2}). Our method of resolution requires the resolution matrices to be constant over time, which means keeping the thermal resistance parameters constant. 
            
            \begin{equation}\label{eq:Reh}
                R_{eh} = \frac{1}{\left(h_c+h_r\right)\cdot S_d}\quad\text{with}\quad
                \begin{dcases}
                    h_c = 4 + 4\times w\\ 
                    h_r = \varepsilon \times 4\sigma {T_{\mathrm{env}}}^3
                \end{dcases}
            \end{equation}

            \noindent
            With,
            $$
            \begin{dcases}
                w &\text{, wind speed (\SI{}{\meter\per\second})}\\ 
                T_{\mathrm{env}} &\text{, environment temperature (\SI{}{\kelvin})}\\ 
                \varepsilon = 0.9 &\text{, corrected hemispheric emissivity}\\  
                \sigma = \SI{5.67e-8}{\watt\per\square\meter\kelvin\tothe{4}} &\text{, Stefan-Boltzmann constant}\\  
            \end{dcases}
            $$
            % paragraph convective_and_radiative_heat_transfer (end)

            \paragraph{Thermal inertia}\mbox{}\\ % (fold)
            \label{par:thermal_inertia}

            The thermal capacities of non-structural elements (windows, insulation, etc.) are assumed to be zero. Thus, only the inertias of the structural elements ($C_e$) for a given material and the internal air volumes ($C_a$) are considered and are expressed in $\SI{}{\joule\per\kelvin}$. The calculation of these two types of inertia is described in equation \eqref{eq:C}. 

            \begin{subequations}\label{eq:C}
                \begin{empheq}[left=\empheqlbrace]{align}
                    C_e &= \rho c_p \times e \times S_d\\ 
                    C_a &= \rho^{\mathrm{air}} c_{p}^\mathrm{air} V^\mathrm{air} 
                \end{empheq}            
            \end{subequations}

            \noindent
            With,
            $$
            \begin{dcases}
                \rho &\text{: density (\SI{}{\kilo\gram\per\cubic\meter})}\\
                c_p &\text{: specific thermal capacity (\SI{}{\joule\per\kilo\gram\kelvin}})\\
                e &\text{: material thickness (\SI{}{\meter}})\\
                S_d &\text{: heat loss surface (\SI{}{\square\meter}})\\
                V^\mathrm{air} &\text{: air volume of building typology (\SI{}{\cubic\meter}})\\
            \end{dcases}
            $$

            To the thermal capacity of the indoor air, we add 12 times this same capacity to take account of the thermal inertia of the internal walls and furniture (\cite{antonopoulos_envelope_1999}).
             
            
            % paragraph thermal_inertia (end)

            \paragraph{Natural and mechanical ventilation}\mbox{}\\ % (fold)
            \label{par:natural_and_mechanical_ventilation}
            
            In ventilation, three phenomena can be distinguished: infiltration of outside air, controlled mechanical ventilation and controlled natural ventilation (by opening windows). Of these elements, and in the context of this modelling, only infiltration can be represented as a thermal resistance (in \SI{}{\kelvin\per\watt}) directly linking the outside temperature to the inside temperature, as a function of the air flow rate \eqref{eq:infiltration} in proportion ($r_{inf}$) of house volume per hour (\SI{}{\per\hour}) (\cite{slama_rt_2016}). Four levels of infiltration can be distinguished: minimal (\SI{0.05}{\per\hour}), low (\SI{0.1}{\per\hour}), medium (\SI{0.2}{\per\hour}), and high (\SI{0.5}{\per\hour}). By default, and in the absence of sufficient data, all types are medium. 

            \begin{equation}\label{eq:infiltration}
                R_{inf} = \frac{3600}{\rho^\mathrm{air}c^\mathrm{air}\times r_{inf}V^\mathrm{air}}
            \end{equation}


            However, controlled (i.e. non-constant) ventilation cannot be modelled as resistance and is therefore considered as internal heat flow in the $\mathbf{u}$ vector \eqref{eq:xtfg}. The thermal flow of natural ventilation is due to the temperature difference between the indoor and outdoor air flows. The mass flow rate being itself a function of the temperature differential (\cite{ashrae_handbook_1997}, \cite{fracastoro_experimental_2002}) \eqref{eq:vent_nat_q}. The discharge coefficient ($C_d$) is considered here to be solely a function of the temperature differential \eqref{eq:vent_nat_c} and the influence of wind speed or the traversing aspect of the dwelling (\cite{ramponi_energy_2014}, \cite{schreck_ventilation_2024}) is neglected. The heat flow is therefore a function of the air mass flow rate \eqref{eq:vent_nat_phi}. Natural ventilation is only used for night-time over-ventilation when the indoor temperature is higher than the set-point cooling temperature and the outdoor temperature. Night-time over-ventilation is deactivated by default (unless otherwise stated) in order to quantify the total cooling needs, even if part of these needs can be met for free by opening the windows.  

            
            \begin{subequations}\label{eq:vent_nat}
                \begin{empheq}[left=\empheqlbrace]{align}
                    q^\mathrm{air} &= \rho^\mathrm{air}~ \frac{S_w}{2} ~ C_d ~ \sqrt{\frac{g\cdot h_w~ |T_i-T_e|}{2T_i}}\label{eq:vent_nat_q}\\ 
                    C_d &= 0.4 + 0.0045\cdot|T_i-T_e|\label{eq:vent_nat_c}\\ 
                    \Phi_{v,\mathrm{nat}} &= q^\mathrm{air}\cdot c_p^\mathrm{air} \left(T_e-T_i \right)\label{eq:vent_nat_phi}
                \end{empheq}            
            \end{subequations}

            \noindent
            With,
            $$
            \begin{dcases}
                \rho^\mathrm{air} &\text{: Air density (\SI{1.2}{\kilo\gram\per\cubic\meter})}\\
                c_p^\mathrm{air} &\text{: Air specific thermal capacity (\SI{1}{\kilo\joule\per\kilo\gram\kelvin})}\\
                g &\text{: Gravitational acceleration (\SI{9.81}{\meter\per\square\second})}\\
            \end{dcases}\quad\text{and}\quad
            \begin{dcases}
                S_w &\text{: Windows surface}\\
                h_w &\text{: Windows height}\\
                T_i &\text{: Internal temperature}\\
                T_e &\text{: Outdoor temperature}\\
            \end{dcases}
            $$


            In addition to natural ventilation, many dwellings have mechanical ventilation. In this case, a fixed flow rate is applied, corresponding to the ventilation obligations in French dwellings (\cite{jorf_arrete_1982}). This mass flow rate is then proportional to the surface area ($S$) of the dwelling \eqref{eq:vent_meca_q}, assuming an average room size of \SI{20}{\square\meter}. Depending on the type of ventilation system, the thermal efficiency varies, and this efficiency is translated as a reduction in the outside air flow entering the dwelling. For example, an airflow of \SI{2}{\kilo\gram\per\second} for an efficiency $\eta$ of 0.7 will be coded as an airflow of \SI{0.6}{\kilo\gram\per\second}. The different systems considered are: simple mechanical ventilation (MV, $\eta=0$), demand controlled ventilation (DCV, $\eta=0.2$), and heat recovery ventilation (HCV, $\eta=0.7$) (\cite{mardiana-idayu_review_2012}, \cite{zhang_novel_2021}). In the case of collective equipment, an efficiency bonus of 0.1 is applied (\cite{bergner_ventilation_2018}). The ventilation heat flow is then expressed in the same way as the heat flow from natural ventilation \eqref{eq:vent_meca_phi}. 

            \begin{subequations}\label{eq:vent_meca}
                \begin{empheq}[left=\empheqlbrace]{align}
                    q^\mathrm{air} &= \frac{\rho^\mathrm{air}}{3600}~\left(24.3 + 0.8\times S\right)\label{eq:vent_meca_q}\\ 
                    \Phi_{v,\mathrm{meca}} &= q^\mathrm{air}\cdot c_p^\mathrm{air} \left(T_e-T_i \right)~\left(1-\eta\right)\label{eq:vent_meca_phi}
                \end{empheq}            
            \end{subequations}

            To avoid retaining too much heat in summer, ventilation systems are equipped with by-passes and can over ventilate at night. So when the inside temperature is higher than the set-point temperature and the outside temperature, the ventilation efficiency is set to 0 and the value of the air mass flow is doubled. 

            % paragraph natural_and_mechanical_ventilation (end)

            \paragraph{External solar gains}\mbox{}\\ % (fold)
            \label{par:external_solar_gains}
            
            Solar radiation plays a very important role in building energy needs. Some solar rays enter the dwelling through the windows, but the majority of the solar radiation heats the external wall of the envelope. Incident solar radiation is separated into two components, which do not behave in the same way: direct solar radiation ($\Phi_{se,\mathrm{direct}}$) and diffuse solar radiation ($\Phi_{se,\mathrm{diffuse}}$). For each orientation of the building (the four directions plus the horizontal), we can calculate the solar power absorbed by the wall surface $S_e$. The proportion of radiation absorbed depends on the albedo of the wall: the closer the albedo is to 1, the less radiation the wall absorbs. In practice, we define four colours: light ($a=0.6$), medium ($a=0.4$), dark ($a=0.2$) and black ($a=0$). 

            \begin{equation}\label{eq:phi_e}
                \Phi_{se} = \left(m\times \Phi_{se,\mathrm{direct}} + \Phi_{se,\mathrm{diffuse}}\right)~ S_e\times a
            \end{equation}

            External masking $m$ blocks direct sunlight when the sun is at an altitude of less than \SI{10}{\degree}. This value may be too high in rural areas but too low in urban areas, and masking can vary greatly due to the proximity of buildings to each other or the presence of trees, etc. 
            % paragraph external_solar_gains (end)

            \paragraph{Internal solar gains}\mbox{}\\ % (fold)
            \label{par:internal_solar_gains}

                In the same way, solar radiation penetrates the glazing (with a surface area $S_w$) of the different orientations and brings a thermal flux directly inside the dwelling. $m$ still represents the distant masking, to which we add the masking of solar protection $m_s$ \eqref{eq:phi_i} represented in \ref{fig:solar_mask_diagram}. 

                \begin{equation}\label{eq:phi_i}
                    \Phi_{se} = \left(m\cdot m_{s,\mathrm{direct}}\times \Phi_{se,\mathrm{direct}} + m_{s,\mathrm{diffuse}}\times\Phi_{se,\mathrm{diffuse}}\right)~ S_w
                \end{equation}

                \begin{figure}[ht]
                \centering
                
                \includegraphics[width=0.32\columnwidth]{figures/solar_mask_direct.png}\hspace{1cm}
                \includegraphics[width=0.32\columnwidth]{figures/solar_mask_diffuse.png}
                
                \caption{\label{fig:solar_mask_diagram} Diagram of solar masks in case of direct and diffuse radiation.}
                    \begin{quote}
                        \vspace{-2mm}
                        \small\noindent
                        \textbf{(left to right)} In the case of direct radiation, the solar masking factor corresponds to the proportion of the window surface reached by the sun's rays. This proportion depends on the length of the mask $e_c$ and the altitude $\alpha$ of the sun in the sky. In the case of diffuse radiation, the masking factor corresponds to the proportion of the sky obscured by the masking element, viewed from the centre of the window. 
                    \end{quote}
                \end{figure}

                The proportion of solar radiation blocked by the masking element depends on the altitude of the sun in the case of direct sunlight \eqref{eq:solar_factor}. In this study, we use different lengths $d$ of the element to block more or less solar radiation. The advantage of this type of system is that it allows solar rays to pass through in winter (when the altitude of the sun is lower), while blocking it in summer, when we want to prevent the dwelling from overheating (\ref{fig:solar_mask_diagram}). 

                \begin{subequations}\label{eq:solar_factor}
                    \begin{empheq}[left=\empheqlbrace]{align}
                        m_{s,\mathrm{direct}} &= 1-\frac{d \times \tan(\alpha) - e_c}{h_w} \label{eq:solar_factor_direct}\\
                        m_{s,\mathrm{diffuse}} &= 1-\frac{1}{\pi/2}~\tan^{-1}\left(\frac{d}{e_c + h_w/2}\right)\label{eq:solar_factor_diffuse}
                    \end{empheq}
                \end{subequations}

                \noindent
                With,
                $$
                \begin{dcases}
                    d&\text{: Solar mask length (\SI{}{\meter})}\\
                    \alpha&\text{: Sun altitude (\SI{}{\degree})}\\
                    h_w&\text{: Windows height (\SI{}{\meter})}\\
                    e_c&\text{: Distance between the solar mask and the window (\SI{}{\meter})}\\
                \end{dcases}
                $$

                % (\ref{fig:solar_mask})

                % \begin{figure}[ht]
                % \centering
                % \includegraphics[width=0.32\columnwidth]{figures/sun_path_Brest_Nice_2022.png}
                % \includegraphics[width=0.32\columnwidth]{figures/direct_solar_factor_masking.png}
                % \includegraphics[width=0.32\columnwidth]{figures/diffuse_solar_factor_masking.png}
                % \caption{\label{fig:solar_mask} Solar path graph and effects of solar protection length on solar transmission factor.}
                %     \begin{quote}
                %         \vspace{-2mm}
                %         \small\noindent
                %         \textbf{(left to right)} Description. \eqref{eq:solar_factor_direct}, \eqref{eq:solar_factor_diffuse}
                %     \end{quote}
                % \end{figure}  
            
            % paragraph internal_solar_gains (end)

            \paragraph{Ground temperature}\mbox{}\\ % (fold)
            \label{par:ground_temperature}
            
            In the majority of thermal building calculation and modelling tools, the ground temperature is considered to be perfectly exogenous, in the same way as the outdoor temperature. The ground temperature is then defined as a function of the annual outdoor temperature data and the thermal parameters of the ground \eqref{eq:kusuda} (\cite{kusuda_earth_1965}). Other research, meanwhile, has worked on new models of soil temperature, based more on heat conservation equations (\cite{badache_new_2016}).  

            \begin{equation}\label{eq:kusuda}
              T(z,t) = \overline{T(0,t)} - \frac{\Delta T(0,t)}{2}~ \exp\left(-\frac{z}{\delta}\right)~ \cos\left(\omega t - \phi_s - \frac{z}{\delta}\right)
            \end{equation}
            
            \noindent
            With,
            $$
            \begin{dcases}
              T(z,t) & \text{: Ground temperature at depth $z$ (in \SI{}{\meter}) and time $t$ (in days (\SI{}{\day}))}\\
              \overline{T(0,t)} & \text{: Annual mean air temperature (in \SI{}{\celsius})}\\
              \Delta T(0,t) & \text{: Annual amplitude of air temperature (in \SI{}{\celsius})}\\
              \delta & \text{: Characteristic depth of soil heat transfer (in \SI{}{\meter})}\\
              \omega=2\pi/365 & \text{: Angular frequency (in \SI{}{\radian\per\day})}\\
              \phi_s & \text{: Phase shift (in \SI{}{\radian})}\\
            \end{dcases}
            $$

            The characteristic depth of soil heat transfer is a function of the thermal diffusivity $\alpha$ of the soil \eqref{eq:diffusivity_1}, which is itself a function of the thermal parameters of the soil \eqref{eq:diffusivity_2}. In our case, we assume the following parameters: $\lambda = \SI{1.5}{\watt\per\meter\kelvin}$, $\rho = \SI{2500}{\kilo\gram\per\cubic\meter}$, and $c_p = \SI{1000}{\joule\per\kilo\gram\kelvin}$. These values are consistent with those found in the literature (\cite{arkhangelskaya_estimating_2018}, \cite{gerard_determination_2020}). Finally, the phase shift is defined as the day of minimum outdoor temperature in the year, expressed in radians \eqref{eq:diffusivity_3}.  
              
            \begin{subequations}\label{eq:diffusivity}
                \begin{empheq}[left=\empheqlbrace]{align}
                    \delta &= \sqrt{\frac{2\alpha}{\omega}} \label{eq:diffusivity_1}\\
                    \alpha &= \frac{\lambda}{\rho c_p} \label{eq:diffusivity_2}\\
                    \phi_s &= \frac{\pi}{365} \times \arg \min_{d \in \llbracket 0,365 \rrbracket} (T(0,d)) \label{eq:diffusivity_3}
                \end{empheq}
            \end{subequations}

            In our case, and in order to take into account the dynamic exchanges between the ground and the modelled buildings, we integrate the ground directly into the RC model as a resistor and a capacitor (\ref{fig:rc_mod}). Modelling the ground by RC analogy is a method used in geothermal research when the aim is to endogenize the thermal responses of the ground (\cite{maestre_new_2015}). Thus, in a semi-infinite 1D soil, differential equation \eqref{eq:rc_soil} is used to define the temperature at a depth $z$ as a function of time. 

            \begin{equation}\label{eq:rc_soil}
              \frac{\partial T_z(t)}{\partial t} = \frac{1}{RC}\left(T_0(t) - T_z(t)\right)
            \end{equation}
            
            \noindent
            With,
            \begin{equation}\label{eq:rc_soil_rc}
              \frac{1}{RC} = \frac{\lambda S}{z} \times \frac{1}{\rho c_p ~ z ~ S} = \frac{\lambda}{\rho c_p ~z^2} = \frac{\alpha}{z^2}
            \end{equation}

            \noindent
            The differential equation \eqref{eq:rc_soil} is solved with the following boundary conditions:
            \begin{itemize}
              \item $T_0(t) = T_e$, where $T_e$ is the outdoor temperature
              \item $T_z(0)$ computed using the \textcite{kusuda_earth_1965} method, defined in equation \eqref{eq:kusuda}. 
            \end{itemize}
            
            To validate the relevance of the RC analogy for the integration of the soil in the thermal model, we compare the model results with data from ERA5 reanalyses (\cite{hersbach_era5_2020}) of soil temperatures at several depths (as available) : $z \in [0.28,1]~\SI{}{\meter}$ and $z\in [1,2.55]~\SI{}{\meter}$ (\ref{fig:ground_temperature}). For ease of access, downloading and manipulation of the data, the ERA5 data is accessed via the open-source Open-Meteo historical weather API (\cite{zippenfenig_open-meteocom_2024}). The selected depths to correspond to the ERA5 data are 2/3 of the depth ranges (i.e. \SI{0.76}{\meter} and \SI{2.03}{\meter}), to match the centre of gravity of the thermal resistances and capacities (proportional to depth). To limit the discrepancies due to initialisation, we carry out the calculation over two years and analyse only the second. The RC model is more accurate than the reference method, with a $\mathrm{RMSE}$ \eqref{eq:rmse} of \SI{1.77}{\celsius} compared with \SI{2.05}{\celsius} for the \SI{76}{\centi\meter} depth, then \SI{1.13}{\celsius} compared with \SI{2.3}{\celsius} for the \SI{2.03}{\meter} depth. 

            \begin{figure}[ht]
                \centering
                
                \includegraphics[width=0.335\columnwidth]{figures/modelling_of_ground_temperature_months_Marseille_2010.png}
                \includegraphics[width=0.655\columnwidth]{figures/timeserie_ground_temperature_Marseille_2010.png}
                
                \caption{\label{fig:ground_temperature} Ground temperature at different depths in 2010 for the Marseille weather.}
                    \begin{quote}
                        \vspace{-2mm}
                        \small\noindent
                        \textbf{(left to right)} Plot of RC modelled soil temperatures as a function of depth for the 12 months of 2010 and outdoor weather corresponding to the city of Marseille in the ERA5 reanalyses. We note a well-known result: from a depth of \SI{10}{\meter}, soil temperature is constant, equal to the average annual surface temperature (\cite{badache_new_2016}, \cite{al-hinti_measurement_2017}). Plot of ground temperatures in 2010 based on ERA5 reanalysis data at \SI{76}{\centi\meter} and \SI{2.03}{\meter} depth. Comparison with the reference model (\cite{kusuda_earth_1965}) and the RC model.
                    \end{quote}
                \end{figure}
            % paragraph ground_temperature (end)
        
        % subsubsection model_computation (end)

    % subsection rc_analogy_and_computation (end)

    \subsection{TABULA typologies} % (fold)
    \label{sub:tabula_typologies}
    
        \subsubsection{Typologies definition} % (fold)
        \label{ssub:typologies_definition}
        
        To avoid simulating all of the 36.8 million dwellings of mainland France (\cite{insee_378_2023}), we carry out thermal modelling on building typologies, architectural archetypes of French buildings. These typologies were defined as part of the European TABULA and EPISCOPE projects, both included in the Intelligent Energy Europe (IEE) Programme and launched in 2009 and 2013 respectively. The TABULA project has made it possible to define building typologies within a common framework for the whole European Union (\cite{loga_typology_2012}). This implies that the method described in this paper can be reproduced for all European countries. The EPISCOPE project then enabled these typologies to be strengthened and to develop initial methods for targeting priority energy renovations in order to reduce heating consumption in European residential buildings (\cite{loga_tabula_2016}). \\

        For mainland France, there are 40 building typologies (\cite{pouget_consultants_batiments_2015}), based on 4 types of building: single-family house (SFH), terraced-house (TH), multi-family house (MFH) (collective flats of less than 10 dwellings), and apartment block (AB)  (of 10 dwellings or more). The typologies also distinguish ten construction periods, coinciding with the major periods in building techniques and construction codes, setting out specific thermal requirements. The first French thermal regulation was the RT1974 (\cite{jorf_arrete_1974}), following the oil crisis of 1973. For each of these 40 types, 3 levels of insulation are also defined: initial (corresponding to the state at the time of construction), standard (corresponding to the requirements of the BBC-Renovation label (\cite{observatoire_bbc_batiments_2019}, \cite{effinergie_label_2023})) and advanced (corresponding to a passive or positive energy building (BEPOS) (\cite{observatoire_bbc_batiments_2018})). The typologies and their main geometric and thermal characteristics are summarised in \ref{tab:tabula_typologies}. Typologies 1, 2, 3 and 4 (corresponding to buildings constructed before 1974) are the only ones without insulation in the initial state, and part of the results will therefore focus on the renovation strategies of these typologies alone. 

        \begin{table}[ht]
            \centering
            \caption{\label{tab:tabula_typologies} Table of EPISCOPE/TABULA typologies at initial level.}
            \small
            \resizebox{\columnwidth}{!}{
            \begin{tabular}{lllcccccccccc}
            \toprule
            Category & Variable & Unit & &  &  &  &  &  & &  &  &  \\
            \midrule
             & Construction period & &   & 1915 & 1949 & 1968 & 1975 & 1982 & 1990 & 2000 & 2006 & 2013 \\
             & & &  1914 & 1948 & 1967 & 1974 & 1981 & 1989 & 1999 & 2005 & 2012 & 2021 \\
             & Insulation standards & boolean &  &  &  &  & \checkmark & \checkmark & \checkmark & \checkmark & \checkmark & \checkmark \\
            \midrule
             &  &  & SFH.01 & SFH.02 & SFH.03 & SFH.04 & SFH.05 & SFH.06 & SFH.07 & SFH.08 & SFH.09 & SFH.10 \\
            \midrule
            Building & Surface & \SI{}{\square\meter} & 88.0 & 86.0 & 72.0 & 119.0 & 130.0 & 144.0 & 97.0 & 111.0 & 95.0 & 103.0 \\
             & Levels & - & 2.0 & 3.0 & 1.5 & 1.5 & 2.0 & 2.0 & 1.0 & 2.0 & 3.0 & 2.0 \\
             & Basement & boolean & \checkmark & \checkmark & \checkmark & \checkmark & \checkmark &  &  & \checkmark & \checkmark & \checkmark \\
             & Converted attic & boolean &  &  &  & \checkmark & \checkmark & \checkmark &  &  &  & \checkmark \\
             & \# semi-detached & - & 0.0 & 0.0 & 0.0 & 0.0 & 0.0 & 0.0 & 0.0 & 0.0 & 0.0 & 0.0 \\
             & Building orientation & - & S & W & W & W & S & S & N & E & E & S \\
             & Windows surface & \SI{}{\square\meter} & 15.0 & 15.0 & 13.0 & 19.0 & 33.0 & 15.0 & 15.0 & 19.0 & 16.0 & 20.0 \\
            Insulation & Wall insulation thickness & \SI{}{\centi\meter} & 0.0 & 0.0 & 0.0 & 0.0 & 4.0 & 8.0 & 8.0 & 10.0 & 0.0 & 10.0 \\
             & Floor insulation thickness & \SI{}{\centi\meter} & 0.0 & 0.0 & 0.0 & 0.0 & 5.0 & 2.0 & 7.0 & 10.0 & 10.0 & 0.0 \\
             & Windows U-value & \SI{}{\watt\per\square\meter\kelvin} & 4.8 & 4.8 & 2.6 & 2.6 & 2.8 & 2.6 & 2.6 & 1.8 & 1.6 & 1.4 \\
             & Roof U-value & \SI{}{\watt\per\square\meter\kelvin} & 1.1 & 1.6 & 2.4 & 1.4 & 0.7 & 0.7 & 0.2 & 0.2 & 0.2 & 0.2 \\
            Energy needs & Heating needs & \SI{}{\kilo\watthour\per\square\meter\year} & 158.0 & 169.0 & 172.0 & 127.0 & 107.0 & 85.0 & 89.0 & 74.0 & 76.0 & 52.0 \\
            \midrule
             &  &  & TH.01 & TH.02 & TH.03 & TH.04 & TH.05 & TH.06 & TH.07 & TH.08 & TH.09 & TH.10 \\
            \midrule
            Building & Surface & \SI{}{\square\meter} & 144.0 & 88.0 & 79.0 & 105.0 & 75.0 & 81.0 & 155.0 & 62.0 & 67.0 & 93.0 \\
             & Levels & - & 2.0 & 2.0 & 2.0 & 2.0 & 3.0 & 2.0 & 2.0 & 2.0 & 1.0 & 1.0 \\
             & Basement & boolean &  & \checkmark &  & \checkmark & \checkmark & \checkmark &  & \checkmark &  &  \\
             & Converted attic & boolean & \checkmark &  &  & \checkmark &  &  &  &  &  &  \\
             & \# semi-detached & - & 1.0 & 2.0 & 1.0 & 2.0 & 2.0 & 1.0 & 2.0 & 2.0 & 1.0 & 1.0 \\
             & Building orientation & - & S & S & W & S & S & E & N & S & W & S \\
             & Windows surface & \SI{}{\square\meter} & 32.0 & 10.0 & 17.0 & 21.0 & 15.0 & 13.0 & 14.0 & 22.0 & 23.0 & 16.0 \\
            Insulation & Wall insulation thickness & \SI{}{\centi\meter} & 0.0 & 0.0 & 0.0 & 0.0 & 4.0 & 7.0 & 8.0 & 10.0 & 0.0 & 12.0 \\
             & Floor insulation thickness & \SI{}{\centi\meter} & 0.0 & 0.0 & 0.0 & 0.0 & 4.0 & 16.0 & 7.0 & 10.0 & 10.0 & 16.0 \\
             & Windows U-value & \SI{}{\watt\per\square\meter\kelvin} & 4.8 & 4.8 & 2.6 & 2.6 & 3.1 & 2.6 & 1.8 & 1.6 & 1.6 & 1.4 \\
             & Roof U-value & \SI{}{\watt\per\square\meter\kelvin} & 1.2 & 2.9 & 1.4 & 3.0 & 0.5 & 0.3 & 0.2 & 0.2 & 0.2 & 0.1 \\
            Energy needs & Heating needs & \SI{}{\kilo\watthour\per\square\meter\year} & 152.0 & 170.0 & 166.0 & 128.0 & 103.0 & 69.0 & 41.0 & 66.0 & 71.0 & 44.0 \\
            \midrule
             &  &  & MFH.01 & MFH.02 & MFH.03 & MFH.04 & MFH.05 & MFH.06 & MFH.07 & MFH.08 & MFH.09 & MFH.10 \\
            \midrule
            Building & Surface & \SI{}{\square\meter} & 213.0 & 394.0 & 397.0 & 460.0 & 179.0 & 851.0 & 682.0 & 497.0 & 594.0 & 539.0 \\
             & Levels & - & 4.0 & 3.0 & 4.0 & 4.0 & 2.0 & 4.0 & 4.0 & 4.0 & 3.0 & 3.0 \\
             & Basement & boolean & \checkmark & \checkmark & \checkmark &  &  & \checkmark &  & \checkmark & \checkmark & \checkmark \\
             & Converted attic & boolean & \checkmark &  &  & \checkmark & \checkmark & \checkmark & \checkmark & \checkmark & \checkmark & \checkmark \\
             & \# semi-detached & - & 0.0 & 2.0 & 2.0 & 0.0 & 0.0 & 0.0 & 0.0 & 1.0 & 0.0 & 0.0 \\
             & Building orientation & - & E & S & S & W & E & W & E & E & S & S \\
             & Windows surface & \SI{}{\square\meter} & 77.0 & 46.0 & 92.0 & 137.0 & 40.0 & 149.0 & 81.9 & 81.0 & 119.0 & 92.0 \\
            Insulation & Wall insulation thickness & \SI{}{\centi\meter} & 0.0 & 0.0 & 0.0 & 4.0 & 4.0 & 8.0 & 10.0 & 10.0 & 10.0 & 21.0 \\
             & Floor insulation thickness & \SI{}{\centi\meter} & 0.0 & 0.0 & 0.0 & 0.0 & 2.0 & 6.0 & 10.0 & 10.0 & 12.0 & 16.0 \\
             & Windows U-value & \SI{}{\watt\per\square\meter\kelvin} & 2.6 & 2.8 & 4.8 & 5.6 & 2.8 & 2.8 & 2.6 & 1.6 & 1.6 & 1.4 \\
             & Roof U-value & \SI{}{\watt\per\square\meter\kelvin} & 1.4 & 2.4 & 2.4 & 0.8 & 0.5 & 0.6 & 0.4 & 0.3 & 0.2 & 0.2 \\
            Energy needs & Heating needs & \SI{}{\kilo\watthour\per\square\meter\year} & 132.0 & 122.0 & 125.0 & 104.0 & 109.0 & 87.0 & 62.0 & 52.0 & 45.0 & 36.0 \\
            \midrule
             &  &  & AB.01 & AB.02 & AB.03 & AB.04 & AB.05 & AB.06 & AB.07 & AB.08 & AB.09 & AB.10 \\
             \midrule
            Building & Surface & \SI{}{\square\meter} & 1713.0 & 754.0 & 1981.0 & 4297.0 & 1227.0 & 3348.0 & 4890.0 & 2344.0 & 4660.0 & 2210.0 \\
             & Levels & - & 7.0 & 7.0 & 10.0 & 6.0 & 7.0 & 5.0 & 8.0 & 7.0 & 6.0 & 6.0 \\
             & Basement & boolean & \checkmark & \checkmark & \checkmark & \checkmark & \checkmark & \checkmark & \checkmark & \checkmark & \checkmark & \checkmark \\
             & Converted attic & boolean & \checkmark & \checkmark & \checkmark & \checkmark & \checkmark & \checkmark & \checkmark & \checkmark & \checkmark & \checkmark \\
             & \# semi-detached & - & 0.0 & 0.0 & 0.0 & 0.0 & 0.0 & 0.0 & 0.0 & 0.0 & 0.0 & 0.0 \\
             & Building orientation & - & W & W & E & W & W & N & S & W & E & N \\
             & Windows surface & \SI{}{\square\meter} & 365.0 & 166.0 & 408.0 & 1146.0 & 260.0 & 306.0 & 725.0 & 215.0 & 800.0 & 340.0 \\
            Insulation & Wall insulation thickness & \SI{}{\centi\meter} & 0.0 & 0.0 & 0.0 & 4.0 & 4.0 & 4.0 & 8.0 & 10.0 & 12.0 & 15.9 \\
             & Floor insulation thickness & \SI{}{\centi\meter} & 0.0 & 0.0 & 0.0 & 0.0 & 5.0 & 8.0 & 8.0 & 10.0 & 16.0 & 12.0 \\
             & Windows U-value & \SI{}{\watt\per\square\meter\kelvin} & 4.8 & 2.8 & 2.6 & 2.6 & 2.8 & 2.6 & 3.3 & 1.6 & 1.6 & 1.4 \\
             & Roof U-value & \SI{}{\watt\per\square\meter\kelvin} & 2.3 & 3.2 & 3.0 & 3.2 & 0.8 & 0.6 & 0.4 & 0.2 & 0.3 & 0.1 \\
            Energy needs & Heating needs & \SI{}{\kilo\watthour\per\square\meter\year} & 161.0 & 109.0 & 108.0 & 94.0 & 105.0 & 64.0 & 66.0 & 41.0 & 35.0 & 22.0 \\
            \bottomrule
            \end{tabular}
            }
            \begin{quote}
                \vspace{-1mm}
                \small\noindent
                Only the main geometric and thermal parameters of the typologies are shown in this table. Other parameters, such as the thickness of walls and their materials, the distinction between glazed surfaces by orientation, etc., are not displayed for the sake of clarity.  
            \end{quote}
        \end{table}

        % subsubsection typologies_definition (end)

        \subsubsection{Typology representation in the French building stock} % (fold)
        \label{ssub:typologies_distribution}
        
        representativite des typologies dans le parc français (\ref{fig:tab_stock})

        \begin{figure}[ht]
            \centering
            % \includegraphics[width=0.99\columnwidth]{figures/bgc_distribution_tabula_buildings_ponderated.png}\\
            \includegraphics[width=0.99\columnwidth]{figures/bgc_distribution_tabula_households_ponderated.png}
            \caption{\label{fig:tab_stock} Distribution of TABULA typologies in the French building stock.}
            % \begin{quote}
            %     \vspace{-2mm}
            %     \small\noindent
            %     \textbf{(top to bottom)} Description.  
            %   \end{quote}
        \end{figure}

        detail de la methode pour la séparation entre SFH et TH

        




        % subsubsection typologies_distribution (end)

        \subsubsection{RC Model verification} % (fold)
        \label{ssub:model_verification}
        
        resultats de comparaisons avec TABULA (\ref{fig:tabula_verif})

        parler des hypothèses propres à la méthodologie de \textcite{pouget_consultants_batiments_2015} et des modifications de behaviour par rapport au conventionnel. 

        \begin{figure}[ht]
            \centering
            \includegraphics[width=0.99\columnwidth]{figures/SFH_TABULA_consumption.png}
            \includegraphics[width=0.99\columnwidth]{figures/TH_TABULA_consumption.png}
            \includegraphics[width=0.99\columnwidth]{figures/MFH_TABULA_consumption.png}
            \includegraphics[width=0.99\columnwidth]{figures/AB_TABULA_consumption.png}
            \caption{\label{fig:tabula_verif} Heating needs comparison with TABULA data.}
                \begin{quote}
                    \vspace{-2mm}
                    \small\noindent
                    \textbf{(top to bottom)} Description. 
                \end{quote}
        \end{figure}  

        comparaison avec \textcite{pomianowski_method_2023} (\ref{fig:model_validation}) (en faire d'autres)

        \begin{figure}[ht]
            \centering
            \includegraphics[width=0.32\columnwidth]{figures/effect_walls_insulation_heating_needs_litterature.png}
            \includegraphics[width=0.32\columnwidth]{figures/blank.png}
            \includegraphics[width=0.32\columnwidth]{figures/blank.png}
            \caption{\label{fig:model_validation} Literature comparison for model validation}
                \begin{quote}
                    \vspace{-2mm}
                    \small\noindent
                    \textbf{(top to bottom)} Description. 
                \end{quote}
        \end{figure}  
        % subsubsection model_verification (end)
    

    % subsection tabula_typologies (end)

    \clearpage
    \subsection{Weather data} % (fold)
    \label{sub:weather_data}

        
        \subsubsection{Historical and projection data} % (fold)
        \label{ssub:historical_data}
        
        \paragraph{Historical data}\mbox{}\\ % (fold)
        \label{par:historical_data}
        
        Once the buildings and their physical and technical characteristics have been defined, the second parameter required for thermal modelling is the basic meteorological data. In this simplified RC model, only outdoor temperatures and direct and diffuse solar radiation are considered. In the more comprehensive building energy simulations, data on relative humidity, wind speed and direction and dew point temperature are also used (\cite{bhandari_evaluation_2012}).\\

        Meteorological reanalysis data from the SAFRAN-ISBA hydro-meteorological model (\cite{habets_safran-isba-modcou_2008}) are used to calibrate the thermal model with past climate data over a continuous and complete spatial and temporal coverage. For ease of access, downloading and manipulation of the data, the ERA5 data is accessed via the GéoSAS SAFRAN API (\cite{bera_geosas_2015}). The SAFRAN reanalysis data are compared in terms of Mean Absolute Error ($\mathrm{MAE}$) \eqref{eq:mae} and Root Mean Square Error ($\mathrm{RMSE}$) \eqref{eq:rmse} with the Météo-France observation data (\ref{tab:rmse_safran_mf}, Appendix \ref{sub:comparison_between_mf_observation_and_era5_reanalysis_data}) to check that there is no systematic bias, at the stations closest to the central prefectures for each of the 8 French climate zones (\ref{ssub:french_climate_zones}). The two compared variables are: the monthly average of daily mean of daily maximal and minimal temperature (MT) in \SI{}{\celsius}, and the monthly cumulative sum of daily global radiation (GR) in \SI{}{\kilo\joule\per\square\centi\meter}. The results are perfectly similar, with differences of around \num{0.6} and \num{2.4}, for temperature and global radiation ranges between 0 and \SI{25}{\celsius} and 10 and \SI{80}{\kilo\joule\per\square\centi\meter} respectively. 

        \begin{subequations}
            \begin{empheq}[left=\empheqlbrace]{align}
                \mathrm{MAE} &= \frac{1}{n}~\sum_{i=1}^n |\hat{y_i}-y_i|\label{eq:mae}\\
                \mathrm{RMSE} &= \sqrt{\frac{1}{n}~\sum_{i=1}^n \left(\hat{y_i}-y_i\right)^2}\label{eq:rmse}
            \end{empheq}
        \end{subequations}
        With,
        $$
        \begin{dcases}
            \hat{y_i}&\text{: the observations values}\\
            y_i&\text{: the reanalysis values}\\
            n&\text{: number of months in the comparison period}\\
        \end{dcases}
        $$

        % \begin{table}[ht]
        %     \caption{\label{tab:rmse_era5_mf} Table of differences between ERA5 reanalysis and Météo-France observation data (mean temperature and global radiation) for the 8 French climate zones.}
        %     \resizebox{\columnwidth}{!}{
        %     \begin{tabular}{lllcccccccc}
        %     \toprule
        %      &&& H1a & H1b & H1c & H2a & H2b & H2c & H2d & H3 \\
            
        %      &unit&& 2010-2023 & 1994-1999 & 2017-2023 & 2009-2023 & 2011-2023 & 1968-2023 & 2004-2017 & 2002-2011 \\
        %     \midrule
        %     MT&\SI{}{\celsius}&$\mathrm{MAE}$ & 0.35 & 0.52 & 0.60 & 0.21 & 0.22 & 0.48 & 1.40 & 0.99 \\
        %     &&$\mathrm{RMSE}$&  0.45 & 0.64 & 0.68 & 0.26 & 0.28 & 0.61 & 1.50 & 1.14 \\
        %     GR&\SI{}{\kilo\joule\per\square\centi\meter} &$\mathrm{MAE}$&  1.50 & 5.22 & 2.43 & 1.70 & 1.47 & 2.06 & 2.83 & 1.70 \\
        %     &&$\mathrm{RMSE}$&  1.94 & 6.37 & 3.68 & 2.27 & 1.90 & 2.76 & 6.10 & 2.10 \\
        %     \bottomrule
        %     \end{tabular}}
        %     \begin{quote}
        %         \vspace{-2mm}
        %         \small\noindent
        %         The time periods under the climate zone names correspond to the time range available for the weather station closest to the location defined for the climate region (i.e. the central prefecture (\ref{fig:zcl})). 
        %     \end{quote}
        % \end{table}

        \begin{table}[ht]
            \caption{\label{tab:rmse_safran_mf} Table of differences between SAFRAN reanalysis and Météo-France observation data (mean temperature MT and global radiation GR) for the 8 French climate zones.}
            \resizebox{\columnwidth}{!}{
            \begin{tabular}{lllcccccccc}
            \toprule
             &&& H1a & H1b & H1c & H2a & H2b & H2c & H2d & H3 \\
            
             &unit&& 2010-2023 & 1994-1999 & 2017-2023 & 2009-2023 & 2011-2023 & 1968-2023 & 2004-2017 & 2002-2011 \\
            \midrule
            MT&\SI{}{\celsius}&$\mathrm{MAE}$ & 0.29 & 0.51 & 1.00 & 0.19 & 0.24 & 0.34 & 1.08 & 0.25 \\
            &&$\mathrm{RMSE}$& 0.35 & 0.55 & 1.09 & 0.24 & 0.32 & 0.43 & 1.19 & 0.35 \\
            GR&\SI{}{\kilo\joule\per\square\centi\meter} &$\mathrm{MAE}$& 4.47 & 5.40 & 2.47 & 6.39 & 2.45 & 3.37 & 6.39 & 3.48 \\
            &&$\mathrm{RMSE}$& 5.64 & 6.76 & 4.02 & 7.63 & 3.19 & 4.52 & 8.71 & 4.52 \\
            \bottomrule
            \end{tabular}}
            \begin{quote}
                \vspace{-2mm}
                \small\noindent
                The time periods under the climate zone names correspond to the time range available for the weather station closest to the location defined for the climate region (i.e. the central prefecture (\ref{fig:zcl})). 
            \end{quote}
        \end{table}

        \paragraph{Projection data}\mbox{}\\ % (fold)
        \label{par:proj_data}

        The climate projections used for this study are 5 combinations of global climate models (GCM) and regional climate models (RCM) from the French Explore2 project (\cite{sauquet_explore2_2021}) and r1i1p1 ensemble (\cite{taylor_cmip5_2010}). All the combinations are listed in the \ref{tab:climate_models}. These were chosen because they presented both temperature projections and solar radiation data. The climate projections were calibrated using the ADAMONT method (\cite{verfaillie_method_2017}).\\

        \begin{table}[ht]
            \caption{\label{tab:climate_models} The Explore2 simulations used in this study.}
            \resizebox{\columnwidth}{!}{
            \begin{tabular}{lllll}
            \toprule
            No. & GCM & RCM  \\
            \midrule
            1   & CNRM-CERFACS-CNRM-CM5 (\cite{voldoire_cnrm-cm51_2013}) & CNRM-ALADIN63 (\cite{nabat_modulation_2020}) \\
            2   & CNRM-CERFACS-CNRM-CM5 (\cite{voldoire_cnrm-cm51_2013}) & MOHC-HadREM3-GA7-07 \\
            3   & ICHEC-EC-EARTH (\cite{hazeleger_ec-earth_2012}) & KNMI-RACMO22E (\cite{van_meijgaard_knmi_2008})\\
            4   & ICHEC-EC-EARTH (\cite{hazeleger_ec-earth_2012}) & MOHC-HadREM3-GA7-05 \\
            5   & MOHC-HadGEM2-ES (\cite{collins_development_2011}) & MOHC-HadREM3-GA7-05 \\
            \bottomrule
            \end{tabular}
            }
        \end{table}

        However, climate data is available on a daily time scale. It is therefore necessary to reconstruct intraday meteorological records. For temperature, the Sin(14R-1) method, an adaptation of the CIBSE method (\cite{petherbridge_weather_1983}, \cite{chow_new_2007}), was used. This method reconstructs hourly temperature data from daily minimum and maximum temperature data. The extreme temperature values are set at sunrise and 3 PM respectively (the details and justification are available in Appendix \ref{sub:details_and_justification_of_sin}). The temperatures at the intermediate times are then the result of a sinusoidal interpolation between the minimum and maximum values, as described in equation \eqref{eq:sin14r1}.

        \begin{equation}\label{eq:sin14r1}
             T(t_i) = \frac{T(t_\mathrm{next})-T(t_\mathrm{prev})}{2} - \left(\frac{T(t_\mathrm{next})-T(t_\mathrm{prev})}{2} \times \cos\left(\frac{\pi\left(t-t_\mathrm{prev}\right)}{t_\mathrm{next}-t_\mathrm{prev}}\right)\right)
        \end{equation} 

        \noindent
        With,
        $$
        \begin{dcases}
            T&\text{: External temperature}\\
            t_i&\text{: time step $i$}\\
            t_\mathrm{prev}&\text{: previous time step where T is defined}\\
            t_\mathrm{next}&\text{: next time step where T is defined}\\
        \end{dcases}
        $$

        By applying this method to the reanalysis data, one can determine the error of this method. The average error is \SI{0.1}{\celsius} and the RMSE \eqref{eq:rmse} is \SI{1.5}{\celsius}, over the period 2000-2020. (Appendix \ref{sub:details_and_justification_of_sin}). \\

        For horizontal solar radiation on an hourly time step, the relationship between the altitude of the sun and global horizontal radiation is used, followed by calibration of the modelled daily mean value to the daily meteorological data, including a corrective factor (Appendix \ref{sub:details_and_justification_of_sin}).

        \begin{equation}\label{eq:solar_eq}
             S(t_i) = \left(\frac{\bar{S_d}}{1/24\times\sum_{t_i \in d} \cos\left(\pi/2-\alpha(t_i)\right)} \times \cos\left(\frac{\pi}{2}-\alpha(t_i)\right)\right)^+
             % {\color{red}1.09}
        \end{equation} 

        \noindent
        With,
        $$
        \begin{dcases}
            S&\text{: Global horizontal radiation at time step $t_i$ of day $d$}\\
            \bar{S_d}&\text{: Daily mean of global radiation}\\
            \alpha&\text{: Sun altitude (in radians)}\\
            (\cdot)^+&\text{: Positive part function}\\
        \end{dcases}
        $$

        Finally, the distribution of global radiation between direct and diffuse radiation is assumed to be constant, in the absence of additional information in the projection data. In past ERA5 data, the average ratio between direct and global radiation is \num{0.749}. (Appendix \ref{sub:details_and_justification_of_sin}). 

        \paragraph{Definition of global warming periods}\mbox{}\\ % (fold)
        \label{par:gw_periods}

        For this study, and to measure the effects of the global warming at different amplitudes, three periods are used:
        \begin{itemize}
            \item Current (or reference) period: 2000-2020
            \item +\SI{2}{\celsius} global warming
            \item +\SI{4}{\celsius} global warming
        \end{itemize}
        To identify the absolute annual mean temperatures corresponding to the +2 and +\SI{4}{\celsius} periods, one can use the data available via the IPCC interactive atlas (\cite{ipcc_climate_2021}, \cite{iturbide_implementation_2022}) and by exporting the Euro-CORDEX (\cite{jacob_euro-cordex_2014}) dataset corresponding to the two levels of warming. The +2 period corresponds to the average of 47 climate models and the +4 period 39 models. By applying a mask over mainland France, we obtain the average annual temperatures in France for these two levels of warming, \SI{11.4}{\celsius} and \SI{13.3}{\celsius} respectively (\ref{fig:gw24}).\\

        \begin{figure}[ht]
            \centering
            \includegraphics[width=0.49\columnwidth]{figures/mean_yearly_temperature_france_gw2deg.png}
            \includegraphics[width=0.49\columnwidth]{figures/mean_yearly_temperature_france_gw4deg.png}
            \caption{\label{fig:gw24} Mean yearly temperature over mainland France for +\SI{2}{\celsius} and +\SI{4}{\celsius} of global warming.}
            \begin{quote}
                \vspace{-2mm}
                \small\noindent
                The data for the two levels of warming come from the Euro-CORDEX data in the IPCC Working Group I interactive atlas. The data has been filtered for mainland France only.
              \end{quote}
        \end{figure}

        These warming temperatures, in the framework of RCP8.5, do not correspond to the same time period for all the climate models used in this study (\ref{tab:climate_models}), because each model has its own dynamics (Appendix \ref{sub:projected_temperature_and_solar_radiation_data}). Thus, the time periods associated with the warming levels of each model are summarised in the \ref{tab:climate_periods} and the maps of mean annual temperatures and annual solar irradiance for these periods are shown in \ref{fig:map_temp_sun}, for climate model 2. 

        \begin{table}[ht]
            \centering
            \caption{\label{tab:climate_periods} Temporal periods of global warming levels in RCP8.5 for the different climate models.}
            % \resizebox{\columnwidth}{!}{
            \small
            \begin{tabular}{ccc}
                \toprule
                Climate model no. & +\SI{2}{\celsius} & +\SI{4}{\celsius} \\
                \midrule
                1 & 2020 -- 2040 & 2059 -- 2079 \\
                2 & 2013 -- 2033 & 2054 -- 2074 \\
                3 & 2017 -- 2037 & 2061 -- 2081 \\
                4 & 2005 -- 2025 & 2047 -- 2067 \\
                5 & 2001 -- 2021 & 2040 -- 2060 \\
                \bottomrule
                \end{tabular}
            % }
            \begin{quote}
                % \vspace{-2mm}
                \small\noindent
                The periods were identified on the basis of the 20-year rolling average of the annual mean temperature records from each of the climate models. In the case of model no. 4 and 5, the period corresponding to +\SI{2}{\celsius} is very close to the reference period, which limits the differences between the current (reference) period and the effects of global warming at +\SI{2}{\celsius}.
            \end{quote}
        \end{table}

        \begin{figure}[ht]
            \centering
            \includegraphics[width=0.32\columnwidth]{figures/map_tas_mod1_2000-2020.png}
            \includegraphics[width=0.32\columnwidth]{figures/map_tas_mod1_2deg.png}
            \includegraphics[width=0.32\columnwidth]{figures/map_tas_mod1_4deg.png}\\
            \includegraphics[width=0.32\columnwidth]{figures/map_rsds_mod1_2000-2020.png}
            \includegraphics[width=0.32\columnwidth]{figures/map_rsds_mod1_2deg.png}
            \includegraphics[width=0.32\columnwidth]{figures/map_rsds_mod1_4deg.png}
            \caption{\label{fig:map_temp_sun} Mean yearly temperature and mean yearly solar irradiance over mainland France for reference period (2000-2020), and difference for periods of +\SI{2}{\celsius} and +\SI{4}{\celsius} of global warming, for the climate model no. 2.}
            \begin{quote}
                \vspace{-2mm}
                \small\noindent
                \textbf{(left to right, top to bottom)} Average of daily mean temperatures between 2000 and 2020 and map of the differences in mean temperatures for the periods with +\SI{2}{\celsius} and +\SI{4}{\celsius} of warming (i.e. 2013-2033 and 2054-2074 in the case of model no. 2). The increase in temperature is relatively uniform over the region, but slightly lower along the Channel. For the daily mean solar radiation, one can recognise the spatial division of the calibration zones, and to some extent the division of the winter climate zones (\ref{fig:zcl}). Model 2 shows an inhomogeneous increase in solar radiation over the region. These patterns differ from one climate model to another (Appendix \ref{sub:projected_temperature_and_solar_radiation_data}). 
              \end{quote}
        \end{figure}

        The inter-annual variability of meteorological data makes the calculation of energy needs for a given building very sensitive. To compensate for this, the French statistics services propose to provide climate-corrected energy consumption figures (\cite{sdes_bilan_2023}). This makes it possible to measure the effect of thermal renovations alone, dissociated from the effects of meteorological variations from one year to the next. However, in the case of this study, we specifically want to look at the interaction between renovations and climate change. The most intuitive way of taking annual variability into account is to carry out several years of energy simulations in order to observe the distribution of needs and their inter-annual variations. \\

        In building energy, the aim is usually to limit calculation time by reducing this meteorological period. It is possible to create an annual meteorological record representative of the entire time period under consideration. This is called a typical meteorological year (TMY) (\cite{wilcox_users_2008}), and its calculation is based on the Sandia method (\cite{hall_generation_1978}). In this study, we prefer to keep the 20 years of meteorological data in order to be able to characterise, in future analyses, the influences of future heat waves, which will be increasingly intense and frequent in the coming years (\cite{ouzeau_heat_2016}).

        % subsubsection historical_data (end)

        \subsubsection{French climate zones} % (fold)
        \label{ssub:french_climate_zones}

        To limit the number of different meteorological chronicles, but still be able to represent the diversity of climates in mainland France, we are basing the study on the 8 French climate zones. This choice also allows us to fit in with the various public policy systems for monitoring and subsidising thermal renovations in France (\cite{anah_aides_2024}). Whether white certificates (\cite{ademe_french_2011}) or energy performance certificates (\cite{ministere_de_la_transition_ecologique_methode_2021}), the underlying methods take into account the climate zone in which the building is located. \\

        For the sake of simplicity, we do not calculate the population-weighted average chronicles of each climate region for each of the meteorological variables (\cite{cros_comparative_2025}). Instead, we consider the central prefecture of the climate region (\ref{fig:zcl}), i.e. the prefecture of the department closest to the geographical centre of the climate region \eqref{eq:pref_centr}. All the climate variables are then identified at the coordinates of this central prefecture. 

        \begin{equation}
            \label{eq:pref_centr}
            p_c = \arg \min_{d \in \mathbf{d}_c} \left(\|\mathbf{c}_c - \mathbf{c}_{dp}\|_2 \right)
        \end{equation}
        With,
        $$
        \begin{dcases}
            \mathbf{d}_c&\text{: list of departments $d$ in climate zone $c$}\\
            \mathbf{c}_c&\text{: geographic coordinates of climate zone $c$ centroid}\\
            \mathbf{c}_{dp}&\text{: geographic coordinates of department $d$ prefecture}\\
        \end{dcases}
        $$

        \begin{figure}[ht]
            \centering
            \includegraphics[width=0.32\columnwidth]{figures/zcl_winter.png}
            \includegraphics[width=0.32\columnwidth]{figures/zcl_summer.png}
            \includegraphics[width=0.32\columnwidth]{figures/zcl.png}
            \caption{\label{fig:zcl} Maps of french climate zones.}
            \begin{quote}
                \vspace{-2mm}
                \small\noindent
                \textbf{(left to right)} Map of \enquote{winter}, \enquote{summer} and combined climate zones. The winter climate zones were defined in a 1988 decree on the thermal characteristics of residential buildings (\cite{jorf_arrete_1988}). Corsica is part of region H3. Summer climate zones were defined in 2000 in parallel with new thermal regulation laws (\cite{jorf_arrete_2000}). The 8 climate zones are a combination of the two zone types (\cite{jorf_arrete_2012}) and the map shows the \enquote{central prefectures} used to define the meteorological data for each zone, as defined in \eqref{eq:pref_centr}. 
                
                 
              \end{quote}
        \end{figure}
        
        % subsubsection french_climate_zones (end)

        % \subsubsection{Standard weather data} % (fold)
        % \label{ssub:standard_weather_data}
        
        

        % The method consists in aggregating the months closest to the monthly cumulative distribution function (CDF) over the entire reference period. Proximity to the general CDF is measured using Finkelstein-Schafer statistics $\mathrm{FS}$ (\cite{finkelstein_improved_1971}) for each meteorological variable. Then aggregated by weighted sum ($\mathrm{WS}$) over all the meteorological parameters. The weightings are adapted from the official method for the variables considered in this study (\cite{wilcox_users_2008}). Thus, the weight of solar radiation is 1/2, that of the mean daily temperature is 1/4 and those of the minimum and maximum daily temperatures are 1/8.  Then, the different months are ranked according to $\mathrm{WS}$ and for each month of the year, the closest is selected. These 12 months of data are then concatenated and the month interfaces are smoothed over 6 hours using a rolling average. 

        % \begin{subequations}
        %     \begin{empheq}[left=\empheqlbrace]{align}
        %         \mathrm{FS}_{i}^m &= |\mathrm{CDF}^{m,\mathrm{lt}}_{i}-\mathrm{CDF}^m_{i}|\label{eq:FS}\\
        %         \mathrm{WS}^m &= \sum_{i} w_{i}\mathrm{FS}_{i}^m\label{eq:WS}
        %     \end{empheq}
        % \end{subequations}
        % With,
        % $$
        % \begin{dcases}
        %     \mathrm{FS}_{i}^m&\text{: Finkelstein-Schafer statistics of month $m$ and climate variable $i$}\\
        %     \mathrm{CDF}^{m,\mathrm{lt}}_{i}&\text{: Long-term CDF of month $m$ and climate variable $i$}\\
        %     \mathrm{CDF}^m_{i}&\text{: CDF of month $m$ and climate variable $i$}\\
        %     \mathrm{WS}^m&\text{: Weighted sum of month $m$}\\
        %     w_{i}&\text{: Climate variables weights}\\
        % \end{dcases}
        % $$

        % Description des modèles et des TMY. 
        
    
    % subsection weather_data (end)

    \subsection{Behaviour definition} % (fold)
    \label{sub:behaviour_definition}


    To distinguish between consumption linked to the intrinsic qualities of the building and that attributable to the behaviour of its occupants, it is important to define a so-called conventional behaviour. This behaviour is derived from national thermal regulations (\cite{ministere_de_la_transition_ecologique_methode_2021-1}). Conventional behaviour defines: the set point temperature for heating and cooling, the internal power released by the occupants and their equipment, and the periods when windows are open (for the calculation of energy consumption).

    \subsubsection{Set-point temperatures} % (fold)
    \label{ssub:set_point_temperature}
    
    In the official Th-BCE method (\cite{ministere_de_la_transition_ecologique_methode_2021-1}), the set temperature for heating is 19°C when the occupants are present in the dwelling and 16°C otherwise. For cooling, the set temperature is 26°C when people are present and 30°C when they are not. The presence period is from 6~p.m. to 10~a.m. on all working days (except Wednesdays, when occupants are present from 1~p.m.). At weekends, it is assumed that occupants are present for the whole day. The evolution of set-point temperatures over a week is shown in \ref{fig:behav}.

    % subsubsection set_point_temperature (end)

    \subsubsection{Internal heat gains} % (fold)
    \label{ssub:internal_heat_gains}
    
    To calculate the internal gains due to residents and their equipment, we distinguish between three elements: gains from lighting ($\Phi_{i,l}$), gains from equipment (excluding lighting) ($\Phi_{i,e}$) and gains from people ($\Phi_{i,p}$) \eqref{eq:Phi_i}. Internal gains depend on the time of day, distinguishing between the presence period ($P_p$) (\ref{ssub:set_point_temperature}) and the sleep period ($P_s$) (from 10~p.m. to 7~a.m.). All these gains are expressed in \SI{}{\watt}.

    \begin{subequations}\label{eq:Phi_i}
        \begin{empheq}[left=\empheqlbrace]{align}
            \Phi_{i,l}(t) &= \label{eq:phi_il}\begin{cases}
                1.4\times S & \text{ if } t \in P_p \text{ and } t \notin P_s\\
                0 & \text{ else.}
            \end{cases}\\
            \Phi_{i,e}(t) &= \label{eq:phi_ie}\begin{cases}
                5.7 \times S & \text{ if } t \in P_p \text{ and } t \notin P_s\\
                1.1 \times S & \text{ else.}
            \end{cases}\\
            \Phi_{i,p}(t) &= \label{eq:phi_ip}\begin{cases}
                90 \times N_{a,\mathrm{eq}}(S) & \text{ if } t \in P_p \text{ and } t \notin P_s\\
                63 \times N_{a,\mathrm{eq}}(S) & \text{ if } t \in P_p \text{ and } t \in P_s\\
                0 & \text{ else.}
            \end{cases}
        \end{empheq}
    \end{subequations}

    With $S$ the house surface and $N_{a,\mathrm{eq}}$ the number of equivalent adults in the household, function of house surface and building type ($t$). $N_{a,\mathrm{eq}}$ is defined in equation \eqref{eq:nae} and represented in function of the house surface in \ref{fig:behav}. Finally, the total internal gain $\Phi_i$ (\ref{fig:behav}) is the sum of the three gains defined above. 

    \begin{subequations}\label{eq:nae}
        \begin{empheq}[left=\empheqlbrace]{align}
            N(S) &= 
            \begin{cases}
                \begin{cases}
                    1 & \text{ if } S<30 \\
                    1.75 - 0.01875 \times (70 - S) & \text{ if } S \in [30,70], \\
                    0.025 \times S & \text{ if } S>70 
                \end{cases} & \text{ if } t \in \{ \mathrm{SFH}, \mathrm{TH}\}\\
                \begin{cases}
                    1 & \text{ if } S<10 \\
                    1.75 - 0.01875 \times (50 - S) & \text{ if } S \in [10,50], \\
                    0.035 \times S & \text{ if } S>50 
                \end{cases} & \text{ if } t \in \{ \mathrm{MFH}, \mathrm{AB}\}
            \end{cases}\\ 
            N_{a,\mathrm{eq}} &= \begin{cases}
                N & \text{ if } N < 1.75\\
                1.75 + 0.3 \times \left(N - 1.75\right) & \text{ else.}
            \end{cases}
        \end{empheq}
    \end{subequations}


    % subsubsection internal_heat_gains (end)

     \begin{figure}[ht]
            \centering
            \includegraphics[width=0.32\columnwidth]{figures/conventionnel_th-bce_2020_heating_cooling_rules.png}
            \includegraphics[width=0.32\columnwidth]{figures/conventionnel_th-bce_2020_nb_adulte_eq.png}
            \includegraphics[width=0.32\columnwidth]{figures/conventionnel_th-bce_2020_internal_gains_rules.png}
            \caption{\label{fig:behav} Conventional behaviour definition graphs.}
            \begin{quote}
                \vspace{-2mm}
                \small\noindent
                \textbf{(left to right)} Evolution of conventional set-point temperatures for heating and cooling over the week. Wednesday can be identified, as its period of presence extends into the afternoon, unlike the other working days. Graph of the relationship between the number of equivalent adults in a dwelling as a function of the surface area (in \SI{}{\square\meter}). The differences between multi-family and single-family dwellings can be explained by the fact that collective housing is generally smaller and lifestyles are denser. In the same way as the set temperatures, the graph shows the change over the week in internal heat gain due to the occupants and their equipment. This heat flow (in \SI{}{\watt}) is noted $\Phi_i$ in the model shown in \ref{fig:rc_mod}. 
                
                 
              \end{quote}
        \end{figure}
    
    % subsection behaviour_definition (end)
% section rc (end)

\clearpage
\section{Interactions between summer and winter comfort}
\label{sec:inter}

    \subsection{Single renovation actions} % (fold)
    \label{sub:single_renovation_actions}
    
    actions monogestes : grande part des travaux de renovations (\cite{ademe_typologie_2019}, \cite{onre_renovation_2022})


    liste des actions et details du choix (\cite{i4ce_trajectoires_2023}, \cite{peuportier_resiliance_2023})

    plus choix des typologies, des climats

        \subsubsection{Roof insulation} % (fold)
        \label{ssub:roof_insulation}
        
            detailler les caractéristiques propre à ce type de rénovation, pourquoi les gens l'ont bcp fait etc. 

            commenter l'interaction (\ref{fig:roof_init})

            \begin{figure}[ht]
                \centering
                \includegraphics[width=0.32\columnwidth]{figures/roof_FR.N.SFH.01.Gen_H1b_conventionnel_th-bce_2020_2000-2020.png}
                \includegraphics[width=0.32\columnwidth]{figures/roof_FR.N.SFH.01.Gen_H2c_conventionnel_th-bce_2020_2000-2020.png}
                \includegraphics[width=0.32\columnwidth]{figures/roof_FR.N.SFH.01.Gen_H3_conventionnel_th-bce_2020_2000-2020.png}\\
                % \includegraphics[width=0.32\columnwidth]{figures/roof_FR.N.TH.01.Gen_H1b_conventionnel_th-bce_2020_2000-2020.png}
                % \includegraphics[width=0.32\columnwidth]{figures/roof_FR.N.TH.01.Gen_H2c_conventionnel_th-bce_2020_2000-2020.png}
                % \includegraphics[width=0.32\columnwidth]{figures/roof_FR.N.TH.01.Gen_H3_conventionnel_th-bce_2020_2000-2020.png}\\
                % \includegraphics[width=0.32\columnwidth]{figures/roof_FR.N.MFH.01.Gen_H1b_conventionnel_th-bce_2020_2000-2020.png}
                % \includegraphics[width=0.32\columnwidth]{figures/roof_FR.N.MFH.01.Gen_H2c_conventionnel_th-bce_2020_2000-2020.png}
                % \includegraphics[width=0.32\columnwidth]{figures/roof_FR.N.MFH.01.Gen_H3_conventionnel_th-bce_2020_2000-2020.png}\\
                % \includegraphics[width=0.32\columnwidth]{figures/roof_FR.N.AB.03.Gen_H1b_conventionnel_th-bce_2020_2000-2020.png}
                % \includegraphics[width=0.32\columnwidth]{figures/roof_FR.N.AB.03.Gen_H2c_conventionnel_th-bce_2020_2000-2020.png}
                % \includegraphics[width=0.32\columnwidth]{figures/roof_FR.N.AB.03.Gen_H3_conventionnel_th-bce_2020_2000-2020.png}
                \caption{\label{fig:roof_init} Effects of roof insulation thickness on energy needs.}
                \begin{quote}
                    \vspace{-2mm}
                    \small\noindent
                    \textbf{(left to right)} Description
                  \end{quote}
            \end{figure}

        % subsubsection roof_insulation (end)

        \subsubsection{Walls insulation} % (fold)
        \label{ssub:walls_insulation}

            description des travaux, ITI car facade ancienne la plupart du temps ? 

            commenter l'interaction (\ref{fig:walls_init})

            \begin{figure}[ht]
                \centering
                \includegraphics[width=0.32\columnwidth]{figures/walls_FR.N.SFH.01.Gen_H1b_conventionnel_th-bce_2020_2000-2020.png}
                \includegraphics[width=0.32\columnwidth]{figures/walls_FR.N.SFH.01.Gen_H2c_conventionnel_th-bce_2020_2000-2020.png}
                \includegraphics[width=0.32\columnwidth]{figures/walls_FR.N.SFH.01.Gen_H3_conventionnel_th-bce_2020_2000-2020.png}\\
                % \includegraphics[width=0.32\columnwidth]{figures/walls_FR.N.TH.01.Gen_H1b_conventionnel_th-bce_2020_2000-2020.png}
                % \includegraphics[width=0.32\columnwidth]{figures/walls_FR.N.TH.01.Gen_H2c_conventionnel_th-bce_2020_2000-2020.png}
                % \includegraphics[width=0.32\columnwidth]{figures/walls_FR.N.TH.01.Gen_H3_conventionnel_th-bce_2020_2000-2020.png}\\
                % \includegraphics[width=0.32\columnwidth]{figures/walls_FR.N.MFH.01.Gen_H1b_conventionnel_th-bce_2020_2000-2020.png}
                % \includegraphics[width=0.32\columnwidth]{figures/walls_FR.N.MFH.01.Gen_H2c_conventionnel_th-bce_2020_2000-2020.png}
                % \includegraphics[width=0.32\columnwidth]{figures/walls_FR.N.MFH.01.Gen_H3_conventionnel_th-bce_2020_2000-2020.png}\\
                % \includegraphics[width=0.32\columnwidth]{figures/walls_FR.N.AB.03.Gen_H1b_conventionnel_th-bce_2020_2000-2020.png}
                % \includegraphics[width=0.32\columnwidth]{figures/walls_FR.N.AB.03.Gen_H2c_conventionnel_th-bce_2020_2000-2020.png}
                % \includegraphics[width=0.32\columnwidth]{figures/walls_FR.N.AB.03.Gen_H3_conventionnel_th-bce_2020_2000-2020.png}
                \caption{\label{fig:walls_init} Effects of walls insulation thickness on energy needs.}
                \begin{quote}
                    \vspace{-2mm}
                    \small\noindent
                    \textbf{(left to right)} Description
                  \end{quote}
            \end{figure}
        
        % subsubsection walls_insulation (end)

        \subsubsection{Floor insulation} % (fold)
        \label{ssub:floor_insulation}

        description des travaux etc.

        detailler les interactions (\ref{fig:floor_init})

        detailler le cas des batiments collectifs 

            \begin{figure}[ht]
                \centering
                \includegraphics[width=0.32\columnwidth]{figures/floor_FR.N.SFH.01.Gen_H1b_conventionnel_th-bce_2020_2000-2020.png}
                \includegraphics[width=0.32\columnwidth]{figures/floor_FR.N.SFH.01.Gen_H2c_conventionnel_th-bce_2020_2000-2020.png}
                \includegraphics[width=0.32\columnwidth]{figures/floor_FR.N.SFH.01.Gen_H3_conventionnel_th-bce_2020_2000-2020.png}\\
                % \includegraphics[width=0.32\columnwidth]{figures/floor_FR.N.TH.01.Gen_H1b_conventionnel_th-bce_2020_2000-2020.png}
                % \includegraphics[width=0.32\columnwidth]{figures/floor_FR.N.TH.01.Gen_H2c_conventionnel_th-bce_2020_2000-2020.png}
                % \includegraphics[width=0.32\columnwidth]{figures/floor_FR.N.TH.01.Gen_H3_conventionnel_th-bce_2020_2000-2020.png}\\
                % \includegraphics[width=0.32\columnwidth]{figures/floor_FR.N.MFH.01.Gen_H1b_conventionnel_th-bce_2020_2000-2020.png}
                % \includegraphics[width=0.32\columnwidth]{figures/floor_FR.N.MFH.01.Gen_H2c_conventionnel_th-bce_2020_2000-2020.png}
                % \includegraphics[width=0.32\columnwidth]{figures/floor_FR.N.MFH.01.Gen_H3_conventionnel_th-bce_2020_2000-2020.png}\\
                \includegraphics[width=0.32\columnwidth]{figures/floor_FR.N.AB.03.Gen_H1b_conventionnel_th-bce_2020_2000-2020.png}
                \includegraphics[width=0.32\columnwidth]{figures/floor_FR.N.AB.03.Gen_H2c_conventionnel_th-bce_2020_2000-2020.png}
                \includegraphics[width=0.32\columnwidth]{figures/floor_FR.N.AB.03.Gen_H3_conventionnel_th-bce_2020_2000-2020.png}
                \caption{\label{fig:floor_init} Effects of floor insulation thickness on energy needs.}
                \begin{quote}
                    \vspace{-2mm}
                    \small\noindent
                    \textbf{(left to right, top to bottom)} Description
                  \end{quote}
            \end{figure}
        
        % subsubsection floor_insulation (end)

        \subsubsection{Color of external surface} % (fold)
        \label{ssub:color_of_external_surface}

            processus physique à décrire

            detailler les interactions (\ref{fig:color_init})

            \begin{figure}[ht]
                \centering
                \includegraphics[width=0.32\columnwidth]{figures/albedo_FR.N.SFH.01.Gen_H1b_conventionnel_th-bce_2020_2000-2020.png}
                \includegraphics[width=0.32\columnwidth]{figures/albedo_FR.N.SFH.01.Gen_H2c_conventionnel_th-bce_2020_2000-2020.png}
                \includegraphics[width=0.32\columnwidth]{figures/albedo_FR.N.SFH.01.Gen_H3_conventionnel_th-bce_2020_2000-2020.png}\\
                % \includegraphics[width=0.32\columnwidth]{figures/albedo_FR.N.TH.01.Gen_H1b_conventionnel_th-bce_2020_2000-2020.png}
                % \includegraphics[width=0.32\columnwidth]{figures/albedo_FR.N.TH.01.Gen_H2c_conventionnel_th-bce_2020_2000-2020.png}
                % \includegraphics[width=0.32\columnwidth]{figures/albedo_FR.N.TH.01.Gen_H3_conventionnel_th-bce_2020_2000-2020.png}\\
                % \includegraphics[width=0.32\columnwidth]{figures/albedo_FR.N.MFH.01.Gen_H1b_conventionnel_th-bce_2020_2000-2020.png}
                % \includegraphics[width=0.32\columnwidth]{figures/albedo_FR.N.MFH.01.Gen_H2c_conventionnel_th-bce_2020_2000-2020.png}
                % \includegraphics[width=0.32\columnwidth]{figures/albedo_FR.N.MFH.01.Gen_H3_conventionnel_th-bce_2020_2000-2020.png}\\
                % \includegraphics[width=0.32\columnwidth]{figures/albedo_FR.N.AB.03.Gen_H1b_conventionnel_th-bce_2020_2000-2020.png}
                % \includegraphics[width=0.32\columnwidth]{figures/albedo_FR.N.AB.03.Gen_H2c_conventionnel_th-bce_2020_2000-2020.png}
                % \includegraphics[width=0.32\columnwidth]{figures/albedo_FR.N.AB.03.Gen_H3_conventionnel_th-bce_2020_2000-2020.png}
                \caption{\label{fig:color_init} Effects of external surface color on energy needs.}
                \begin{quote}
                    \vspace{-2mm}
                    \small\noindent
                    \textbf{(left to right)} Description
                  \end{quote}
            \end{figure}
        
        % subsubsection color_of_external_surface (end)

        \subsubsection{Ventilation efficiency} % (fold)
        \label{ssub:ventilation_efficiency}

            décrire les types de ventilation et leurs principales différences 

            mentionner le by pass et surventilation nocturne 

            detailler les interactions (\ref{fig:ventil_init})

            \begin{figure}[ht]
                \centering
                \includegraphics[width=0.32\columnwidth]{figures/ventilation_FR.N.SFH.01.Gen_H1b_conventionnel_th-bce_2020_2000-2020.png}
                \includegraphics[width=0.32\columnwidth]{figures/ventilation_FR.N.SFH.01.Gen_H2c_conventionnel_th-bce_2020_2000-2020.png}
                \includegraphics[width=0.32\columnwidth]{figures/ventilation_FR.N.SFH.01.Gen_H3_conventionnel_th-bce_2020_2000-2020.png}\\
                % \includegraphics[width=0.32\columnwidth]{figures/ventilation_FR.N.TH.01.Gen_H1b_conventionnel_th-bce_2020_2000-2020.png}
                % \includegraphics[width=0.32\columnwidth]{figures/ventilation_FR.N.TH.01.Gen_H2c_conventionnel_th-bce_2020_2000-2020.png}
                % \includegraphics[width=0.32\columnwidth]{figures/ventilation_FR.N.TH.01.Gen_H3_conventionnel_th-bce_2020_2000-2020.png}\\
                % \includegraphics[width=0.32\columnwidth]{figures/ventilation_FR.N.MFH.01.Gen_H1b_conventionnel_th-bce_2020_2000-2020.png}
                % \includegraphics[width=0.32\columnwidth]{figures/ventilation_FR.N.MFH.01.Gen_H2c_conventionnel_th-bce_2020_2000-2020.png}
                % \includegraphics[width=0.32\columnwidth]{figures/ventilation_FR.N.MFH.01.Gen_H3_conventionnel_th-bce_2020_2000-2020.png}\\
                % \includegraphics[width=0.32\columnwidth]{figures/ventilation_FR.N.AB.03.Gen_H1b_conventionnel_th-bce_2020_2000-2020.png}
                % \includegraphics[width=0.32\columnwidth]{figures/ventilation_FR.N.AB.03.Gen_H2c_conventionnel_th-bce_2020_2000-2020.png}
                % \includegraphics[width=0.32\columnwidth]{figures/ventilation_FR.N.AB.03.Gen_H3_conventionnel_th-bce_2020_2000-2020.png}
                \caption{\label{fig:ventil_init} Effects of ventilation system on energy needs.}
                \begin{quote}
                    \vspace{-2mm}
                    \small\noindent
                    \textbf{(left to right)} Description
                \end{quote}
            \end{figure}
        
        % subsubsection ventilation_efficiency (end)

        \subsubsection{Shading of glazed surfaces} % (fold)
        \label{ssub:shading_of_glazed}

        description des travaux, cf schéma 

        detailler les interactions (\ref{fig:shading_init})

        importance des volets et des comportements associés 

        \begin{figure}[ht]
            \centering
            % \includegraphics[width=0.32\columnwidth]{figures/shading_FR.N.SFH.01.Gen_H1b_conventionnel_th-bce_2020_2000-2020.png}
            % \includegraphics[width=0.32\columnwidth]{figures/shading_FR.N.SFH.01.Gen_H2c_conventionnel_th-bce_2020_2000-2020.png}
            % \includegraphics[width=0.32\columnwidth]{figures/shading_FR.N.SFH.01.Gen_H3_conventionnel_th-bce_2020_2000-2020.png}\\
            \includegraphics[width=0.32\columnwidth]{figures/shading_FR.N.TH.01.Gen_H1b_conventionnel_th-bce_2020_2000-2020.png}
            \includegraphics[width=0.32\columnwidth]{figures/shading_FR.N.TH.01.Gen_H2c_conventionnel_th-bce_2020_2000-2020.png}
            \includegraphics[width=0.32\columnwidth]{figures/shading_FR.N.TH.01.Gen_H3_conventionnel_th-bce_2020_2000-2020.png}\\
            % \includegraphics[width=0.32\columnwidth]{figures/shading_FR.N.MFH.01.Gen_H1b_conventionnel_th-bce_2020_2000-2020.png}
            % \includegraphics[width=0.32\columnwidth]{figures/shading_FR.N.MFH.01.Gen_H2c_conventionnel_th-bce_2020_2000-2020.png}
            % \includegraphics[width=0.32\columnwidth]{figures/shading_FR.N.MFH.01.Gen_H3_conventionnel_th-bce_2020_2000-2020.png}\\
            % \includegraphics[width=0.32\columnwidth]{figures/shading_FR.N.AB.03.Gen_H1b_conventionnel_th-bce_2020_2000-2020.png}
            % \includegraphics[width=0.32\columnwidth]{figures/shading_FR.N.AB.03.Gen_H2c_conventionnel_th-bce_2020_2000-2020.png}
            % \includegraphics[width=0.32\columnwidth]{figures/shading_FR.N.AB.03.Gen_H3_conventionnel_th-bce_2020_2000-2020.png}
            \caption{\label{fig:shading_init} Effects of solar protection length over windows on energy needs.}
            \begin{quote}
                \vspace{-2mm}
                \small\noindent
                \textbf{(left to right)} Description
            \end{quote}
        \end{figure}
        
        % subsubsection shading_of_ (end)
    % subsection characterisation_of_single_renovation_actions (end)    

    \subsection{Energy needs for TABULA typologies} % (fold)
    \label{sub:energy_needs_for_tabula_typologies}

        % \subsubsection{Single Family House (SFH)} % (fold)
        % \label{ssub:sfh}

        % detailler les résultats (\ref{fig:sfh_needs})
        
        % \begin{figure}[ht]
        %     \centering
        %     \includegraphics[width=0.49\columnwidth]{figures/typology_energy_needs_SFH_H1b_2000-2020.png}
        %     \includegraphics[width=0.49\columnwidth]{figures/typology_energy_needs_SFH_H3_2000-2020.png}
        %     \caption{\label{fig:sfh_needs} Heating and cooling needs for SFH typologies.}
        %     \begin{quote}
        %         \vspace{-2mm}
        %         \small\noindent
        %         \textbf{(left to right)} Description
        %     \end{quote}
        % \end{figure}

        % % subsubsection sfh (end)

        % \subsubsection{Terraced House (TH)} % (fold)
        % \label{ssub:th}

        % detailler les résultats (\ref{fig:th_needs})

        % \begin{figure}[ht]
        %     \centering
        %     \includegraphics[width=0.49\columnwidth]{figures/typology_energy_needs_TH_H1b_2000-2020.png}
        %     \includegraphics[width=0.49\columnwidth]{figures/typology_energy_needs_TH_H3_2000-2020.png}
        %     \caption{\label{fig:th_needs} Heating and cooling needs for TH typologies.}
        %     \begin{quote}
        %         \vspace{-2mm}
        %         \small\noindent
        %         \textbf{(left to right)} Description
        %     \end{quote}
        % \end{figure}
        
        % % subsubsection th (end)

        % \subsubsection{Multi Family House (MFH)} % (fold)
        % \label{ssub:mfh}
        
        % detailler les résultats (\ref{fig:mfh_needs})

        % \begin{figure}[ht]
        %     \centering
        %     \includegraphics[width=0.49\columnwidth]{figures/typology_energy_needs_MFH_H1b_2000-2020.png}
        %     \includegraphics[width=0.49\columnwidth]{figures/typology_energy_needs_MFH_H3_2000-2020.png}
        %     \caption{\label{fig:mfh_needs} Heating and cooling needs for MFH typologies.}
        %     \begin{quote}
        %         \vspace{-2mm}
        %         \small\noindent
        %         \textbf{(left to right)} Description
        %     \end{quote}
        % \end{figure}

        % % subsubsection mfh (end)

        % \subsubsection{Appartment Block (AB)} % (fold)
        % \label{ssub:ab}

        % detailler les résultats (\ref{fig:ab_needs})
        
        % \begin{figure}[ht]
        %     \centering
        %     \includegraphics[width=0.49\columnwidth]{figures/typology_energy_needs_AB_H1b_2000-2020.png}
        %     \includegraphics[width=0.49\columnwidth]{figures/typology_energy_needs_AB_H3_2000-2020.png}
        %     \caption{\label{fig:ab_needs} Heating and cooling needs for AB typologies.}
        %     \begin{quote}
        %         \vspace{-2mm}
        %         \small\noindent
        %         \textbf{(left to right)} Description
        %     \end{quote}
        % \end{figure}

        % subsubsection ab (end)
        
    % subsection energy_needs_for_tabula_typologies (end)
% section inter (end)

\clearpage
\section{Climate impact on optimal renovations}
\label{sec:opti}

    \subsection{Optimal energy efficiency} % (fold)
    \label{sub:optimal_energy_efficiency}
    
    % subsection optimal_energy_efficiency (end)

    \subsection{Optimal economic efficiency} % (fold)
    \label{sub:optimal_economic_efficiency}
    
    % subsection optimal_economic_efficiency (end)
% section opti (end)

% \clearpage
% \section{Generalisation across the whole French building stock}
% \label{sec:generalisation}
    
%     \subsection{Building stock calibration} % (fold)
%     \label{sub:calibration}
    
%     utilisation de la base DPE (statistiques)

%     description de la représentatitivté de la base DPE (BDNB) (\ref{fig:epc})

%     \begin{figure}[ht]
%         \centering
%         \includegraphics[width=0.32\columnwidth]{figures/carte_repr_maison_dpe_nblog-BDNB.png}
%         \includegraphics[width=0.32\columnwidth]{figures/carte_repr_appartement_dpe_nblog-BDNB.png}
%         \includegraphics[width=0.32\columnwidth]{figures/DPE_distribution_dpe_france.png}
%         \caption{\label{fig:epc} Representativeness of EPC data in terms of number of dwellings and in energy performance labels.}
%         \begin{quote}
%             \vspace{-2mm}
%             \small\noindent
%             \textbf{(left to right)} Representativeness maps for single-family homes (SFH + TH) and multi-family homes (MFH + AB), and comparison of the distribution of energy performance labels in France (\cite{onre_parc_2022}) and in the BDNB database. Data from the BDNB (\cite{cstb_base_2024}), version 2023-11-a, aggregating data from the DPE database and the property database (\cite{ademe_donnees_2024}, \cite{cerema_documentation_2024}). 
%         \end{quote}
%     \end{figure}
%     % subsection calibration (end)

%     \subsection{Projection of heating and cooling needs} % (fold)
%     \label{sub:evolution_des_besoins_énergétiques}
    
%     % subsection evolution_des_besoins_énergétiques (end)

%     \subsection{Projection of annual and instant energy demand} % (fold)
%     \label{sub:évoltuion_des_consommations}
    
%     % subsection évoltuion_des_consommations (end)
% % section generalisation (end)

\clearpage
\section{Discussion}
\label{sec:disc}
% section disc (end)

\clearpage
\section{Conclusion}
\label{sec:conclu}
% section conclu (end)



\clearpage
\printbibliography


\appendix

\clearpage
\section{Appendix} % (fold)
\label{sec:appendix}
    
    \subsection{Computation matrices and vectors definitions} % (fold)
    \label{sub:computation_matrices_and_vectors_definitions}
    Vecteurs $\mathbf{x}$ des températures inertielles ($[10\times 1]$) et vecteur $\mathbf{u}$ des températures et flux variables ($[15\times 1]$)

    \begin{equation}
          \begin{dcases}
            \mathbf{x} = [T_i, T_{w_0}, T_{w_1}, T_{w_2}, T_{w_3}, T_c, T_u, T_f, T_d, T_g]^\top \\
            \mathbf{u} = [T_e, \phi_{sue}, \phi_{sui}, \phi_{sw_0e}, \phi_{sw_0i}, \phi_{sw_1e}, \phi_{sw_1i}, \phi_{sw_2e}, \phi_{sw_2i}, \phi_{sw_3e}, \phi_{sw_3i}, \phi_{hc}, \phi_{i}, \phi_{v\mathrm{meca}}, \phi_{v\mathrm{nat}}]^\top
          \end{dcases}
        \end{equation}

        Matrices de calcul $\mathbb{A}$ ($[10\times 10]$) et $\mathbb{B}$ ($[10\times 15]$)
          % \resizebox{\columnwidth}{!}{
        % \resizebox{\columnwidth}{!}{$
        \begin{equation}\label{eq:rcmatrix}
        \mathbb{A} = 
        \begin{bmatrix}
          a_{00} & a_{01} & a_{02} & a_{03} & a_{04} & a_{05} &        & a_{07} &        &       \\
          a_{10} & a_{11} &        &        &        &        &        &        &        &       \\
          a_{20} &        & a_{22} &        &        &        &        &        &        &       \\
          a_{30} &        &        & a_{33} &        &        &        &        &        &       \\
          a_{40} &        &        &        & a_{44} &        &        &        &        &       \\
          a_{50} &        &        &        &        & a_{55} & a_{56} &        &        &       \\
                 &        &        &        &        & a_{65} & a_{66} &        &        &       \\
          a_{70} &        &        &        &        &        &        & a_{77} & a_{78} & a_{79}\\
                 &        &        &        &        &        &        & a_{87} & a_{88} & a_{89}\\
                 &        &        &        &        &        &        & a_{97} & a_{98} & a_{99}\\
        \end{bmatrix}
        \end{equation}


        \begin{equation}
          \mathbb{B} = \begin{bmatrix}
  b_{00} &        & b_{02} &        & b_{04} &        & b_{06} &        & b_{08} &        & b_{010}& b_{011}& b_{012}& b_{013}& b_{014}\\
  b_{10} &        &        & b_{13} &        &        &        &        &        &        &        &        &        &        &       \\
  b_{20} &        &        &        &        & b_{25} &        &        &        &        &        &        &        &        &       \\
  b_{30} &        &        &        &        &        &        & b_{37} &        &        &        &        &        &        &       \\
  b_{40} &        &        &        &        &        &        &        &        & b_{49} &        &        &        &        &       \\
  b_{50} & b_{51} &        &        &        &        &        &        &        &        &        &        &        &        &       \\
  b_{60} & b_{61} &        &        &        &        &        &        &        &        &        &        &        &        &       \\
         &        &        &        &        &        &        &        &        &        &        &        &        &        &       \\
         &        &        &        &        &        &        &        &        &        &        &        &        &        &       \\
  b_{90} &        &        &        &        &        &        &        &        &        &        &        &        &        &       \\
\end{bmatrix}
        \end{equation}


    % subsection computation_matrices_and_vectors_definitions (end)

    \subsection{Effects of time resolution on energy and power needs} % (fold)
    \label{sub:effects_of_time_resolution_on_energy_and_power_needs}
    
    % subsection effects_of_time_resolution_on_energy_and_power_needs (end)

    \subsection{U-values comparison with TABULA typologies} % (fold)
    \label{sub:u_values_comparison_with_tabula_typologies}
    
    % subsection u_values_comparison_with_tabula_typologies (end)

    \subsection{Comparison between Météo-France observation and ERA5 reanalysis} % (fold)
    \label{sub:comparison_between_mf_observation_and_era5_reanalysis_data}
    
    \begin{figure}[ht]
        \centering
        \includegraphics[width=0.49\columnwidth]{figures/comparison_TM_MF_ERA5.png}
        \includegraphics[width=0.49\columnwidth]{figures/comparison_GLOT_MF_ERA5.png}
        \caption{\label{fig:comparison_mf} Comparison of mean temperature and global radiation between ERA5 reanalysis and Météo-France observations.}
        \begin{quote}
            \vspace{-2mm}
            \small\noindent
            \textbf{(left to right)} Versus plot of the monthly average of the daily mean of maximum (TX) and minimum (TN) daily temperatures for climatic regions H1b and H3, the coldest and warmest regions respectively in mainland France. Versus plot of monthly cumulative daily horizontal solar radiation. The time periods shown in the legend correspond to the time range available for the weather station closest to the location defined for the climate region 
          \end{quote}
    \end{figure}

    % subsection comparison_between_météo_france_observation_and_era5_reanalysis_data (end)

    \subsection{Projected temperature and solar radiation data} % (fold)
    \label{sub:projected_temperature_and_solar_radiation_data}
    
    % subsection projected_temperature_and_solar_radiation_data (end)

    \subsection{Hourly weather data construction and validation} % (fold)
    \label{sub:details_and_justification_of_sin}
    
    % subsection details_and_justification_of_sin (end)
    

    \subsection{Single renovation actions on initial typologies} % (fold)
    \label{sub:single_renovation_actions_on_initial_typologies}
    
    % subsection single_renovation_actions_on_initial_typologies (end)

    \subsection{Sensitivity analysis} % (fold)
    \label{sub:sensitivity_analysis}
        
        \subsubsection{Internal set-point temperature} % (fold)
        \label{ssub:internal_set_point_temperature}
        
        % subsubsection internal_set_point_temperature (end)

        \subsubsection{External temperature} % (fold)
        \label{ssub:external_temperature}
        
        % subsubsection external_temperature (end)

        \subsubsection{Global solar radiation} % (fold)
        \label{ssub:global_solar_radiation}
        
        % subsubsection global_solar_radiation (end)

    % subsection sensitivity_analysis (end)

% section appendix (end)


\end{document}